\documentclass[12pt, a4paper]{article}
\usepackage[slovene]{babel}
\usepackage[utf8]{inputenc}
\usepackage[T1]{fontenc}
\usepackage{amsmath, amssymb, fullpage, bbm, float, caption, graphicx}

\newcommand{\A}{
	\ensuremath{\mathbf{A}}
}

\begin{document}

\begin{center}
\textsc{Vi\v sje ra\v cunske metode}\\
\textsc{2012/13}\\[0.5cm]
\textbf{7. naloga -- MPA in Schmidtov algoritem}
\end{center}
\begin{flushright}
\textbf{Jo\v ze Zobec}
\end{flushright}

\section{Schmidtov algoritem in matri\v cno produktni nastavki}

\subsection{Uvod}
Navodila so opisana na spletu. Za uvod bomo zato najprej pogledali, kako do Heisenbergovega hamiltoniana.

Imamo Heisenbergov hamiltonian za spinsko verigo (v odsotnostni magnetnega polja), ki ga poznamo že iz
drugega poglavja (in raznih drugih kurzov):
\[
	H = -J\sum_j \vec{S}_j \cdot \vec{S}_{j+1} = -J \sum_j \hat{h}_{j,j+1}.
\]

Hamiltonian smo razcepili na vsoto dvodelčnih hamiltonianov, ki se glase
\[
	\hat{h}_{j,j+1} = S_j^x S_{j+1}^x + S_j^y S_{j+1}^y + S_j^z S_{j+1}^z.
\]

Seveda, nam ta oblika ne diši preveč, spomnimo pa se, da spinski operatorji zadoščajo
grupi SU(2) in zanje lahko poiščemo operatorje dviganja in spuščanja:
\begin{align*}
	S_j^+ &= S_j^x + iS_j^y,\\
	S_j^- &= S_j^x - iS_j^y.
\end{align*}

To uporabimo na dvodelčnem hamiltonianu $\hat{h}_{j,j+1}$ in dobimo
\[
	\hat{h}_{j,j+1} = \frac{1}{2}\big(S^+_jS^-_{j+1} + S^-_jS^+_{j+1} + 2S^z_jS^z_{j+1}\big).
\]

Izračunajmo $\hat{h}^{(2)}_{1,2}$ v spinski bazi:
\[
	\hat{h}^{(2)}_{1,2} = \frac{1}{2 \cdot 2}\begin{bmatrix}
		1 & 0 & 0 & 0 \\
		0 &-1 & 2 & 0 \\
		0 & 2 & -1& 0 \\
		0 & 0 & 0 & 1
	\end{bmatrix} = \frac{1}{4} h^{(2)}_{1,2}
\]

Zgornji indeks `2' v oklepaju, nas spomni, da to velja samo, kadar imamo trivialno verigo z zgolj dvema
\v clenoma.

Na tem mestu lahko brez izgube splošnosti redefiniramo $J \to J' = 4J$, zato, da se znebimo tistih
nadle\v znih polovic in nam ostane samo matrika $h_{j,j+1}$. 

Imamo anti-feromagnetno sklopitev, tj. J < 0. Nobenega razloga ni, zakaj ne bi bil kar $J = -1$.

Popoln hamiltonian za več delcev je enak vsoti Kroneckerjevih produktov večih takih matrik
$h^{(2)}_{1,2}$ z identitetami. Seveda deluje le za sodo število delcev (tj. za verige dolžine $
= 2m$, $m \in \mathbb{N}$).

Skratka, $n$-delčni hamiltonian $H^{(n)}$ dobimo kot
\begin{equation}
	H^{(n)} = \sum_{j = 1}^{n - 1} h^{(n)}_{j,j+1} = \sum_{j = 1}^{2m - 1} \mathbbm{1}_{2^{j-1}}
	\otimes h^{(2)}_{1,2} \otimes \mathbbm{1}_{2^{2m - j - 1}}
	\label{hamilton}
\end{equation}

Torej, na\v so matriko $4 \times 4$ vozimo po produktu identitet s korakom $2$. Za primer, ko $m = 2$,
tj. verigo dol\v zine $4$, to pomeni:
\[
	H^{(4)} = \underbrace{h^{(2)}_{1,2}\otimes\mathbbm{1}_4}_{h^{(4)}_{1,2}} +
		\underbrace{\mathbbm{1}_2\otimes h^{(2)}_{1,2}\otimes\mathbbm{1}_2}_{h^{(4)}_{2,3}} +
		\underbrace{\mathbbm{1}_4\otimes h^{(2)}_{1,2}}_{h^{(4)}_{3,4}}
\]

Vendar kot vidimo, ta formula velja samo za odprte robne pogoje. Za periodi\v cne robne pogoje
moramo razcepiti matriko $h^{(2)}_{1,2}$ na Kroneckerjev produkt matrik dimenzije $2 \times 2$, saj
moramo za periodi\v cne robne pogoje dobiti \v clen oblike `$B \otimes \mathbbm{1}_4 \otimes A$',
pri \v cemer smo predpostavili, da je $h^{(2)}_{1,2} = A \otimes B$.

Razcep naredimo nekako takole: opazimo, da sta diagonalna bloka v matriki $h^{(2)}_{1,2}$ enaka
$\sigma^z$ in $-\sigma^z$, izvendiagonalna bloka pa sta $2\sigma^+$ in $2\sigma^-$, tj. blo\v cno

\[
	h^{(2)}_{1,2} = \begin{bmatrix}
		\sigma^z & 2\sigma^- \\
		2\sigma^+ & -\sigma^z
	\end{bmatrix} = 
		\begin{bmatrix}
			\sigma^z & 0 \\
			0 & -\sigma^z
		\end{bmatrix} + 2\begin{bmatrix}
			0 & \sigma^- \\
			\sigma^+ & 0
		\end{bmatrix}.
\]

Takoj opazimo, da je prva matrika enaka $\sigma^z \otimes \sigma^z$. Drugo matriko pa moramo
razcepiti \v se naprej:
\[
	\sigma^\pm = \frac{1}{2}\big(\sigma^x \pm i\sigma^y\big),
\]

\[
	2\begin{bmatrix}
		0 & \sigma^- \\
		\sigma^+ & 0
	\end{bmatrix} = 2 \cdot \frac{1}{2} \begin{bmatrix}
		0 & \sigma^x - i\sigma^y \\
		\sigma^x + i\sigma^y & 0
	\end{bmatrix} = \begin{bmatrix}
		0 & \sigma^x \\
		\sigma^x & 0
	\end{bmatrix} + \begin{bmatrix}
		0 & -i\sigma^y \\
		i \sigma^y & 0
	\end{bmatrix} = \sigma^x \otimes \sigma^x + \sigma^y \otimes \sigma^y.
\]

Torej velja
\[
	h^{(2)}_{1,2} = \sum_{\lambda \in \{x,y,z\}} \sigma^\lambda \otimes \sigma^\lambda
\]

Za prej\v snji zgled, ko smo imeli verigo dol\v zine $4$, bi torej potrebovali \v se \v clen
$h^{(4)}_{4,1}$. Tega lahko sedaj zapi\v semo kot
\[
	h^{(4)}_{4,1} = \sigma^x\otimes\mathbbm{1}_4\otimes\sigma^x +
		\sigma^y\otimes\mathbbm{1}_4\otimes\sigma^y +
		\sigma^z\otimes\mathbbm{1}_4\otimes\sigma^z
\]

Za verigo dol\v zine $n = 2m$ je to
\begin{equation}
	h^{(n)}_{n,1} \equiv \sum_{\lambda \in \{x,y,z\}} \sigma^\lambda \otimes
	\mathbbm{1}_{4^{m - 1}} \otimes \sigma^\lambda
	\label{popravek}
\end{equation}

Hamiltonian s periodi\v cnimi robnimi pogoji, $\tilde{H}^{(n)}$, je torej enak tistemu iz
ena\v cbe~\eqref{hamilton}, le da mu pri\v stejemo popravek~\eqref{popravek}

\begin{equation}
	\tilde{H}^{(n)} \equiv H^{(n)} + h^{(n)}_{n,1}
\end{equation}

\subsection{Rezultati}

Schmidtov algoritem sem preveril z matri\v cno-produktnimi nastavki, ki so mi preverjeno delovali
(tj. ustrezno mno\v zene matrike $\mathbf{A}_{s_n}^{(n)}$ so mi dale nazaj prave koeficiente).

\subsubsection{Schmidtov algoritem}

Najprej bom predstavil rezultate, dobljene iz Heisenbergovega hamiltoniana, ki sem ga omenil v
prej\v snjem poglavju. Grafa na sl.~\ref{1-kompakten} in~\ref{1-nekompakten} prikazujeta entropijo
za simetri\v cno biparticijo v odvisnosti od dol\v cine verige.

\begin{figure}[H]\centering
	\input{1-kompakten.tex}
	\caption{Kompaktna izbira po\v casi v povpre\v cju premo nara\v sca z dol\v cino verige.
		Vidimo, da je za periodi\v cne robne pogoje entropija ve\v cja.}
	\label{1-kompakten}
\end{figure}

\begin{figure}[H]\centering
	\input{1-nekompakten.tex}
	\caption{Nara\v scanje je sedaj bolj linearno kot prej. Vidimo, da imata premici isti naklon,
		razlikujeta se zgolj za aditivno konstanto.}
	\label{1-nekompakten}
\end{figure}

Poglejmo si tudi kako se graf spreminja v odvisnosti od razmerja particij v biparticiji za periodi\v cen in
neperiodi\v cen hamiltonian.

\begin{figure}[H]\centering
	\input{1-periodicni-aji.tex}
	\caption{Za periodi\v cni problem graf spominja na parabolo, simetri\v cno okrog $|A|/n = 1:2$. Vidimo, da je
		tam entropija najve\v cja. V resnici to ni parabola, saj je vrh preve\v c zaobljen.}
	\label{1-periodicni-aji}
\end{figure}

\begin{figure}[H]\centering
	\input{1-neperiodicni-aji.tex}
	\caption{Tudi v tem primeru dobimo v povpre\v cju parabolo -- dobimo dve paraboli. Lahko si mislimo, da imamo
		"`lihe"' in "`sode"' biparticije. Vse sode le\v zijo na eni paraboli, vse lihe pa na drugi.}
	\label{1-neperiodicni-aji}
\end{figure}

Dokaz, da tisto res niso kvadratne parabole sem naredil tako, da sem fital na verigo dol\v zine 12. Za limitne primere na
robovih, tj. $|A|/n \in \{0,1\}$ sem res lahko vzel $0$, kar potrjuje tudi algoritem.

\v Ce je tisti izraz parabola, mora biti cela dru\v zina oblike $f(x;\alpha) = \alpha x(x-1)$, kjer je $x = |A|/n$. Ugibamo,
da je kvadratna potenca premajhna in poizkusimo tudi s kvarti\v cnim nastavkom, $g(x;\alpha,\beta) = -\alpha(x-1/2)^4 + \beta$.
Ta se res bolj obnese.

\begin{figure}[H]\centering
	\input{1-ni-parabola-a.tex}
	\caption{Vidimo, da odvisnost od razmerja velikosti particij ni kvadratna, je pa verjetno kvarti\v cna.}
	\label{1-ni-parabola-a}
\end{figure}

\subsubsection{Matri\v cno produktni nastavki}

Na\v se stanje $|\psi\rangle$ predstavimo s stolpcem
\[
	|\psi\rangle = \sum_{s_1,\ldots,s_n \in \{\uparrow,\downarrow\}} C_{s_1,s_2,\ldots,s_n} |s_1,\ldots,s_n\rangle
\]

kjer je $C_{s_1,\ldots,s_n} = \langle s_1,s_2,\ldots,s_n|\psi\rangle = C_{\underline{s}}$. Vsak koeficient
$C_{\underline{s}}$ lahko predstavimo kot produkt matrik $\mathbf{A}^{(j)}_{s_j}$. Npr. koeficient
$C_{\uparrow\downarrow\downarrow\downarrow}$ zapi\v semo kot
\[
	C_{\uparrow\downarrow\downarrow\downarrow} = \A^{(1)}_\uparrow\A^{(2)}_\downarrow\A^{(3)}_\downarrow\A^{(4)}_\downarrow
\]

Prav tako, morajo biti diagonalne matrike $\lambda^{(j)}$ enake Schmidtovim koeficientom za razcep $j = |A|$. To se izka\v ze,
da je res: entropija prepletenosti mora biti enaka. Za periodi\v cni hamiltonian dobimo namre\v c iste kvarti\v cne parabole,
kar vidimo na grafu~\ref{2-periodicni-aji}.

\begin{figure}[H]\centering
	\input{2-periodicni-aji.tex}
	\caption{Dobimo isto kvarti\v cno parabolo, kot v prej\v snjem primeru, vendar ne moremo definirati skrajnih
		robnih to\v ck, tj. $|A| = 0$ in $|A| = n$, kar je sicer v redu, saj brez biparticije ni entropije
		prepletenosti.}
	\label{2-periodicni-aji}
\end{figure}

Ker mora teorija delovati tudi za komplekesen primer, sem naredil isto za  naklju\v cno izbran normiran vektor. Njegove
komponente so naklju\v cno izbrana Gaussovo porazdeljena \v stevila s te\v zi\v s\v cem v ni\v cli. Preverjal sem, ali
dobim res pravi vektor nazak z rekonstrukcijo in ali dimenzija vektorja vpliva na natan\v cnost. To sem delal tako, da
sem meril kot med vhodnim in izhodnim vektorjem -- $\alpha$ in razliko njunih norm $\Delta N$. Ugotovil sem, da je napaka
neodvisna od dimenzije vektorja in da je $\alpha \sim 10^{-8}$ in $\Delta N \sim 10^{-16}$. Torej je metoda zelo natan\v cna.

Graf~\ref{2-gaus} prikazuje kak\v sne entropije nam vrne Gaussov naklju\v cni vektor za razli\v cne izbire biparticije.
Pri\v cakovali bi, da je brez reda.

\begin{figure}[H]\centering
	\input{2-gaus.tex}
	\caption{Entropija prepletenosti je za naklju\v cni Gaussov vektor vse prej kot naklju\v cna. Spet vidimo
		red. Ta graf bolj spominja na kvadrati\v cno parabolo kot prej\v snji, vendar vidimo, da raste bolj polo\v zno
		in potem zelo hitro spremeni odvod, tako da to ni to. Opazimo lahko, da je naklju\v cno izbrano stanje bolj
		prepleteno, kot osnovno stanje.}
	\label{2-gaus}
\end{figure}

\end{document}
