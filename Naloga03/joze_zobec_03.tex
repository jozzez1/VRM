\documentclass[a4 paper, 12pt]{article}
\usepackage[slovene]{babel}
\usepackage[utf8]{inputenc}
\usepackage[T1]{fontenc}
\usepackage[small, width=0.8\textwidth]{caption}
\usepackage[pdftex]{graphicx}
\usepackage{amssymb, amsmath, fullpage, float, hyperref}

\begin{document}

\begin{center}
\textsc{Višje računske metode}\\
\textsc{2012/13}\\[0.5cm]
\textbf{3. naloga -- Simplektični integratorji}
\end{center}
\begin{flushright}
\textbf{Jože Zobec}
\end{flushright}

\section{Uvod}

V tej nalogi smo se posvetili simplektičnim integratorjem. Tako ime imajo, ker ohranjajo simplektično strukturo, kjer
jo drugi, npr. implicitna Eulerjeva metoda, ali pa tudi vsem dobro znana metoda Runge-Kutta reda 4 ({\tt RK4}). Vse te
metode diferencialno enačbo aproksimirajo z nekimi končnimi diferencami ali pa se je lotijo reševati iterativno.

Simplektični integrator uporablja Liouvillov propagator, ki ga razcepimo s Trotter-Suzukijevem razcepom. Izračunamo
ga lahko do poljubnega reda natančno s pomočjo ti. BCS formul.

Naš hamiltonian je

\[
	H = \underbrace{\frac{1}{2}(p_1 + p_2)^2}_T + \underbrace{\frac{1}{2}(q_1^2 + q_2^2) + \lambda q_1^2q_2^2}_V.
\]

Anharmonski člen $\lambda$ nam problem zakomplicira tako, da diferencialne enačbe s standarnimi enačbami ne moremo
reševati na dolgi rok brez adaptivnega koraka. Zaradi zlobnega potenciala potrebujemo izjemno majhe korake proč od
minimuma potenciala. Za take primere so simplektični integratorji še posebej dobri.

Druga naloga je zahtevala numeričen eksperiment, s katerim bi preverili ekviparticijski teorem, ki pravi da na vsako
prostostno stopnjo dobimo $k_BT/2$ energije, oz. $\langle p_1^2 \rangle = \langle p_2^2 \rangle$, kar bi moralo še
bolj veljati za dolge čase.

\section{Izračun}

Glavno gonilo simplektične integracije je razcep Liouvillovega propagatorja

\begin{equation}
	U(\cdot, H) = \exp\Big(\tau\{\cdot, H\}\Big),
\end{equation}

kjer $\{A,B\}$ pomeni poissonov oklepaj. V aproksimaciji v prvem redu je ta propagator kar

\begin{equation}
	U \approx \exp\Big(\tau\{\cdot, T\}\Big)\exp\Big(\tau\{\cdot, V\}\Big),
\end{equation}

vendar pa ta shema ni simetrična. Lahko izberemo simetrizirano shemo, s pomočjo katere lahko sestavimo tudi vse tiste,
ki so natančnosti $\mathcal{O}(\tau^{2n})$.

\begin{equation}
	\mathtt{S_2} (\tau) = \exp\bigg(\frac{\tau}{2}\{\cdot, T\}\bigg) \exp\bigg(\tau\{\cdot, V\}\bigg)
		\exp\bigg(\frac{\tau}{2}\{\cdot, T\}\bigg),
\end{equation}

čeprav lahko tudi tu z vso pravico zamenjamo $V$ in $T$. Ta shema ima natančnost $\mathcal{O}(\tau^2)$. Skalirnim
faktorji pred $\tau$ v eksponentu (koeficienti) morajo biti za lihe rede natančnosti kompleksni.

Za naš problem imamo vektor $\vec{x}_t = (q_1, q_2, p_1, p_2)_t$. V vektor $\vec{x}_{t+1}$ pridemo z večkratnim
zaporednim delovanjem $\exp(c\tau\{\cdot,V\})$ in $\exp(c\tau\{\cdot,T\})$ na njegove komponente. Za naš primer se te
spremenijo tako:

\begin{align}
	\exp\Big(c\tau\{\cdot, T\}\Big)q_k &= q_k + c\tau p_k, \notag \\
	\exp\Big(c\tau\{\cdot, T\}\Big)p_k &= p_k, \notag \\
	\exp\Big(c\tau\{\cdot, V\}\Big)q_k &= q_k, \notag \\
	\exp\Big(c\tau\{\cdot, V\}\Big)p_k &= p_k - c\tau q_1 (1 + 2\lambda q_2^2) \delta_{1,k} - c\tau q_2
		(1 + 2\lambda q_1^2) \delta_{2,k},
\end{align}

kjer je $c$ lahko v splošnem nek kompleksni koeficient.

\section{Rezultati}

V prvi nalogi je bilo treba zgolj izračunati trajektorije za različne $\lambda$ in preveriti ohranitev energije za
različne integratorje -- ${\tt RK4}$, ${\mathtt S_2}$, ${\mathtt S_4}$ \ldots

Sproti sem še narisal spreminjanje $\langle p_1^2 \rangle$ in $\langle p_2^2 \rangle$ od koraka do koraka. Barva
predstavlja jakost potenciala. Izmed grafov četvorčkov imajo grafi označeni "`povprečje"' narobe označeni osi:
abscisa je namreč $t$ in ordinatna os je $\langle p^2 \rangle$.

Pričnimo z asimetričnim simplektičnim integratorjem reda $\mathcal{O}(\tau)$.

\begin{figure}[H]
   \begin{center}
      \input{n3-L0-s1-t1.tex}
   \end{center}
   \vspace{-20pt}
   \caption{To je natančnost reda $\tau$, ki je enak $0.01$. Kot vidimo se energija ohranja približno najmanj na 6 decimalnih
	mest. Orbita je lepa elipsa, saj je $\lambda = 0$.}
   \label{fig:pic-L0}
   \vspace{-10pt}
\end{figure}

\begin{figure}[H]
   \begin{center}
      \input{n3-L100-s1-t1.tex}
   \end{center}
   \vspace{-20pt}
   \caption{Isto kot prej, le da je tu $\lambda = 1$,  zato je orbita nekaj čudnega. Energija je določena s praktično
	 isto natančnostjo kot prej.}
   \label{fig:pic-L100}
   \vspace{-10pt}
\end{figure}

\begin{figure}[H]
   \begin{center}
      \input{n3-L1000-s1-t1.tex}
   \end{center}
   \vspace{-20pt}
   \caption{V tem primeru je $\lambda = 10$. Orbita praktično zapolni prostor, ki ji je energijsko na voljo, opazimo pa tudi to,
	da ekviparticijski izrek tu že kar dobro drži. Natančnost je malce padla -- pet decimalnih mest.}
   \label{fig:pic-L1000}
   \vspace{-10pt}
\end{figure}

Orbita je tu praktično zapolnila energijsko možen prostor in na prvi pogled je videti kot zmazek. Verjetno smo prekoračili
kritično energijo, pri kateri postane motnja $\lambda$ tako velika, da ne dobimo več lepih periodičnih tirnic.

Poglejmo še simetrični integrator reda $\mathcal{O}(\tau^2)$:

\begin{figure}[H]
   \begin{center}
      \input{n3-L1000-s2-t1.tex}
   \end{center}
   \vspace{-20pt}
   \caption{Simetrični integrator ${\mathtt S_2}$, $\tau = 0.01$, $\lambda = 10$. Kot vidimo je natančnost tu bistveno večja,
	saj imamo 10 decimalnih mest, torej je res za oceno $\mathcal{O}(\tau^2)$. Izračun povprečij ne da neke razlike glede
	na asimetrično shemo.}
   \label{fig:pic-L1000}
   \vspace{-10pt}
\end{figure}

Shema četrtega reda -- ${\mathtt S_4}$:

\begin{figure}[H]
   \begin{center}
      \input{n3-L1000-s4-t1.tex}
   \end{center}
   \vspace{-20pt}
   \caption{${\mathtt S_4}$, $\tau = 0.01$, $\lambda = 10$. Vidimo, da je energija še natančnejša od metod nižjega reda.
	Povprečji kvadratov gibalnih količin se že od začetka držita bolj skupaj.}
   \label{fig:pic-L1000}
   \vspace{-10pt}
\end{figure}

Primerjajmo sedaj to z enim najbolj popularnih integratorjev, {\tt RK4}. Da bo pošteno, preverimo še delovanje ${\mathtt S_4}$
na $\lambda = 0$:

\begin{figure}[H]
   \begin{center}
      \input{n3-L0-s4-t1.tex}
   \end{center}
   \vspace{-20pt}
   \caption{Shema ${\mathtt S_4}$ za $\tau = 0.01$ in $\lambda = 0$. Natančnost je v najslabšem primeru 22 decimalnih mest.}
   \label{fig:pic-L0}
   \vspace{-10pt}
\end{figure}

\begin{figure}[H]
   \begin{center}
      \input{n3-L0-s3-t1.tex}
   \end{center}
   \vspace{-20pt}
   \caption{Shema ${\tt RK4}$, za $\tau = 0.01$ in $\lambda = 0$. Uporabil sem že pripravljeno metodo iz knjižnjice
	{\tt GSL}, pri čemer sem zahteval fiksen korak. Takoj nas zbode v oči, da je orbita napačna in neperiodična.
	Krivina tira je prevelika, da bi bila elipsa, zato vedno zgreši stičišče in nato precedira okrog. Energija se
	ohranja presenetljivo dobro, vendar jo, kot vidimo, monotono nosi proti nižjim natančnostim. Tako imamo od začetnih
	22 decimalnih mest na koncu le še 18 dobrih.}
   \label{fig:pic-L0}
   \vspace{-10pt}
\end{figure}

Poglejmo si še kaj se zgodi za velike $\lambda$. Vzeli bomo kot prej, $\lambda = 10$.

\begin{figure}[H]
   \begin{center}
      \input{n3-L1000-s3-t1.tex}
   \end{center}
   \vspace{-20pt}
   \caption{Integrator je Runge-Kutta reda 4, $\tau = 0.01$, $\lambda = 10$. Objokujem tiste, ki morajo za tak primer
	uporabljati ${\tt RK4}$. Rešitev je zelo slaba. O orbiti sicer ne vemo nič, za naše potrebe se od simplektičnih
	orbit ne razlikuje, čeprav vidimo, da je že ob krajšem času bolj zdivjala od simplektične. Vidimo, pa da je energija
	skoraj vse, samo tisto, kar bi morala biti, ni. Natančnost je kvečjemu na eno decimalno mesto. Vidimo, pa da
	ekviparticijski izrek tu drži "`bolje"'.}
   \label{fig:pic-L1000}
   \vspace{-10pt}
\end{figure}

\pagebreak

\subsection{Veljavnost ekviparticijskega izreka}

To sem naredil kar tako, da sem drsel čez parameter $\lambda$ in izračunal logaritem razmerja povprečij kvadratov gibalnih
količin, torej $\log\langle p_1^2 \rangle / \langle p_2^2 \rangle$. Uporabil sem integrator ${\mathtt S_4}$ in računal
povprečje do časa $T = 10000$. Zaradi konsistence sem spet vzel $\tau = 0.01$.

\begin{figure}[H]
   \begin{center}
      \input{scan-t1.tex}
   \end{center}
   \vspace{-20pt}
   \caption{Vidimo, da kljub začetni nevaljavnosti
	ekviparticijskega izreka, ta prične veljati za recimo $\lambda = 3$, z višanjem pa še bolj, ker je gibanje čedalje
	bolj stohastično in prične spomoinjati na idealni plin.}
      \label{fig:pic-L}
   \vspace{-10pt}
\end{figure}

\end{document}

