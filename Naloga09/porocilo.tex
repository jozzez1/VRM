\documentclass[12pt, a4 paper]{article}
\usepackage[slovene]{babel}
\usepackage[T1]{fontenc}
\usepackage[utf8]{inputenc}
\usepackage{amsmath, amssymb, bbm}

\newcommand{\e}{
	\ensuremath{\mathrm{e}}
}

\newcommand{\ch}{
	\operatorname{ch}
}

\newcommand{\sh}{
	\operatorname{sh}
}

\renewcommand{\d}{
	\ensuremath{\mathrm{d}}
}

\begin{document}

\section{Uvod}

Valovna funkcija pri \v casu $t$ se zapi\v se lahko kot
\[
	|\psi(t)\rangle = \e^{-itH} |\psi(t = 0)\rangle
\]

\v Ce zelimo isto stanje dobiti pri neki inverzni temperaturi $\beta$, je pravilni pristop, da $\beta$ obravnavamo
kot negativni imaginarni \v cas, tj.

\[
	|\psi\rangle_{\beta} = |\psi(-i\beta)\rangle = \e^{(-i\beta)(-i H)}|\psi(t = 0)\rangle = \e^{-\beta H}
		|\psi(\beta=0)\rangle.
\]

Iz prej\v snje naloge se spomnimo, da lahko vse ugotovimo iz dvodel\v cnega hamiltoniana, $h^{(2)}_{1,2}$. V soglasju s tem
bomo definirali dvodel\v cni propagator:
\[
	U^{(2)}(\beta) = \exp\Big({-\beta h^{(2)}_{1,2}}\Big) = \exp\bigg(-\beta\sum_{\lambda \in \{x,y,z\}}\sigma^\lambda
		\otimes \sigma^\lambda\bigg).
\]

Ker matrike $\sigma^\lambda \otimes \sigma^\lambda$ med seboj komutirajo $\forall \lambda$, lahko izraz zapi\v semo kot
produkt treh matrik (vi\v sji \v cleni razvoja BCH so enaki ni\v c)

\begin{equation}
	U^{(2)}(\beta) \equiv \prod_{\lambda \in \{x,y,z\}} \exp\big(-\beta \sigma^\lambda \otimes \sigma^\lambda\big).
\end{equation}

Razvoj lepo konvergira in velja

\[
	\Big(\sigma^\lambda \otimes \sigma^\lambda\Big)^n = \left\{
		\begin{array}{cc}
			\mathbbm{1}, & n = 2m, m \in \mathbb{N} \\
			\sigma^\lambda \otimes \sigma^\lambda, & n \in 2m - 1, m \in \mathbb{N}
		\end{array}
	\right.
\]

Ko to upo\v stevamo, ugotovimo, da je
\[
	U^{(2)}(\beta) = \e^{-\beta h^{(2)}_{1,2}} = \e^{\beta}\begin{bmatrix}
		\e^{-2\beta} & 0 & 0 & 0\\
		0 & \ch(2\beta) & -\sh(2\beta) & 0 \\
		0 & -\sh(2\beta) & \ch(2\beta) & 0 \\
		0 & 0 & 0 & e^{-2\beta}
	\end{bmatrix}
\]

Spinska korelacija je definirana kot
\[
	C_{jk} \sim \langle \sigma_j^z \sigma_k^z \rangle = \langle \psi| \sigma_j^z \sigma_k^z |\psi \rangle,
\]

kjer je o\v citno
\[
	\sigma_j^z \equiv \mathbbm{1}_{2^{j-1}} \otimes \sigma^z \otimes \mathbbm{1}_{2^{n-j}}.
\]

Na\v s na\v crt je, da bomo uporabili ena\v cbe (76), (77), (78) in (79) iz skripte. Torej potrebujemo matrike
$\tilde{\mathbf{A}}^{(j)}_{s_j}$, ki generirajo $\psi^*_{s_1,s_2,\ldots,s_n}$. Trdimo, da $\tilde{\mathbf{A}}^{(j)}_{s_j} =
\big(\mathbf{A}^{(j)}_{s_j}\big)^*$, kar lahko zlahka doka\v zemo:

\begin{align*}
	\psi_{s_1,s_2,\ldots,s_n} &= \mathbf{A}^{(1)}_{s_1} \mathbf{A}^{(2)}_{s_2} \cdots \mathbf{A}^{(n)}_{s_n} \\
	\psi^*_{s_1,s_2,\ldots,s_n} &= \Big(\mathbf{A}^{(1)}_{s_1} \cdots \mathbf{A}^{(n)}_{s_n}\Big)^*
\end{align*}

Na tem mestu bi sicer lahko rekli, da je $\psi^*_{s_1,s_2,\ldots,s_n}$ skalar in da je kompleksno konjugiranje skalarja
enakovredno hermitskemu konjugiranju (tj. hermitiranju) vendar tega ne smemo, saj bi nam to obrnilo vrstni red mno\v zenja,
\v cesar pa ne smemo! Mno\v zimo namre\v c urejeno z "`leve proti desni"'. Tj. zgornji izraz enostavno zapi\v semo kot

\begin{align*}
	\psi^*_{s_1,s_2,\ldots,s_n} &= \big(\mathbf{A}^{(1)}_{s_1}\big)^* \big(\mathbf{A}^{(2}_{s_2}\big)^* \cdots
		\big(\mathbf{A}^{(n)}_{s_n}\big)^* \\
		&= \tilde{\mathbf{A}}^{(1)}_{s_1} \cdots \tilde{\mathbf{A}}^{(n)}_{s_n}.
\end{align*}

Ta produkt je \v ze urejen po spinih, vidimo da je res
\[
	\tilde{\mathbf{A}}^{(j)}_{s_j} \equiv \big(\mathbf{A}^{(j)}_{s_j}\big)^* = \mathbf{A}^{(j)}_{s_j}{}^*
\]

Na\v s operator $O_{s,s'}$ je kar $\sigma^z$. Iz njega moramo dobiti operatorje $\mathbf{V}^{(j)}$ iz ena\v cbe (77). Dobiti
jih je precej enostavno. Imamo vsoto po \v stirih elementih. Od tega sta dva enaka ni\v c, saj $\sigma^z_{\uparrow\downarrow}
= \sigma^z_{\downarrow\uparrow} = 0$. Torej

\begin{align}
	\mathbf{V}^{(j)} &= \sigma^z_{\downarrow\downarrow} \mathbf{A}^{(j)}_{\downarrow}{}^* \otimes
		\mathbf{A}^{(j)}_{\downarrow} + \sigma^z_{\uparrow\uparrow} \mathbf{A}^{(j)}_{\uparrow}{}^*
		\otimes \mathbf{A}^{(j)}_{\uparrow} = \notag \\
		&= \mathbf{A}^{(j)}_{\downarrow}{}^* \otimes \mathbf{A}^{(j)}_{\downarrow} - \mathbf{A}^{(j)}_\uparrow{}^*
		\otimes \mathbf{A}^{(j)}_\uparrow.
\end{align}

Analogno lahko vidimo, da $\mathbf{L} \equiv \mathbf{T}^{(1)}$ in $\mathbf{R} \equiv \mathbf{T}^{(n)}$ in

\begin{align}
	\mathbf{T}^{(j)} = \mathbf{A}^{(j)}_\downarrow{}^* \otimes \mathbf{A}^{(j)}_\downarrow +
		\mathbf{A}^{(j)}_\uparrow{}^* \otimes \mathbf{A}^{(j)}_\uparrow .
\end{align}

Torej je $\langle \sigma^z_j \sigma^z_k \rangle$

\[
	\langle \psi |\sigma^z_j \sigma^z_k | \psi\rangle = \mathbf{T}^{(1)} \cdots \mathbf{T}^{(j-1)} \mathbf{V}^{(j)}
		\mathbf{T}^{(j+1)} \cdots \mathbf{T}^{(k-1)} \mathbf{V}^{(k)} \mathbf{T}^{(k+1)}\cdots \mathbf{T}^{(n)},
\]

pri \v cemer moramo biti na robne pogoje, tj. $2 \leq j < k \leq n-1$. Nimamo periodi\v cnih robnih pogojev, torej nimamo
prave translacijske simetrije tudi za naklju\v cno \v zrebane vektorje. Vendar pa moramo ta izraz izpovpre\v citi po
nekem \v casu, tj.

\[
	C_{ij}(\beta) \equiv \lim_{t \to \infty} \frac{1}{t}\int_0^t \d \tau \langle \psi_\beta(\tau)|\sigma_j^z\sigma_k^z
		|\psi_\beta(\tau)\rangle
\]

\v Ce bomo za\v cetno stanje dovolj ohladili ($\beta \to \infty$) bomo slej, ko prej dosegli osnovno stanje (fizikalna
interpretacija je o\v citna, matemati\v cno pa je tudi res, saj je po absolutni vrednosti za na\v s hamiltonian vedno
najve\v cja lastna vrednost osnovnega stanja, kar pomeni, da v bistvu i\v s\v cemo osnovno stanje s poten\v cno metodo).

Korelacijska funkcija osnovnega stanja je potem kar

\[
	C_{ij}^\infty = C_{ij}(\beta \to \infty) = \lim_{t \to \infty} \frac{1}{t} \int_0^t \d\tau \lim_{\beta \to \infty}
		\langle\psi_\beta(\tau)|\sigma_j^z\sigma_k^z|\psi_\beta(\tau)\rangle.
\]

Vrstni red limitiranja je tu pomemben. Najprej moramo limitirati $\beta$, saj moramo na\v se naklju\v cno stanje najprej
ohladiti in \v sele potem povpre\v citi po \v casu. Stanje je treba sproti \v se normirati, tj. za splo\v sni primer se izraz
glasi

\begin{equation}
	C_{ij}^\infty = C_{ij}(\beta \to \infty) = \lim_{t \to \infty}\frac{1}{t}\int_0^t \d\tau \lim_{\beta \to \infty}
		\frac{\langle\psi_\beta(\tau)|\sigma^z_j\sigma^z_k|\psi_\beta(\tau)\rangle}
		{\langle\psi_\beta(\tau)|\psi_\beta(\tau)\rangle}
\end{equation}

Vendar, pa je osnovno stanje lastno stanje in se ne spreminja s \v casom, torej ni treba povpre\v citi po \v casu.

\v Ce se sedaj posvetimo \v se nalogi z dvema zvezdicama, vidimo, da moramo vzeti za za\v cetno stanje bodisi
\[
	|\psi\rangle = |\underbrace{\uparrow\uparrow\ldots\uparrow}_{m}\underbrace{\downarrow\downarrow\ldots\downarrow}_{m}
		\rangle,
\]
bodisi
\[
	|\psi\rangle = |\underbrace{\downarrow\downarrow\ldots\downarrow}_{m}\underbrace{\uparrow\uparrow\ldots\uparrow}_{m}
		\rangle,
\]
kjer je na\v sa veriga dolga $n = 2m$ spinov. Kateri indeks to predstavlja, \v ce spine uredimo v binarno (npr. $\uparrow$
predstavlja $1$, in $\downarrow = 0$?). Naj se to zgodi pri indeksu $\Lambda$.

Definirajmo
\begin{multline}
	S_+^{(m)} = \overbrace{11\ldots1}^{m}\overbrace{00\ldots0}^{m}{}_2 = 2^{2m-1} + 2^{m-2} + \ldots + 2^m = \\
		2^m(1 + 2 + 4 + \ldots + 2^{m-1}) = 2^m (2^m - 1).
\end{multline}

Zgornji izraz v oklepaju bi potemtakem ustrezal
\begin{equation}
	S_-^{(m)} = \overbrace{00\ldots0}^{m}\overbrace{11\ldots1}^{m}{}_2 = 1 + 2 + \ldots + 2^{m-1} = 2^m-1.
\end{equation}

Imamo torej dve mo\v znosti, za izbiro osnovnega stanja, ki je
\[
	\sum_{\underline{s} = \downarrow}^{\uparrow} |\psi_{\underline{s}}\rangle \delta_{\underline{s}, \Lambda},
\]
kjer je $\Lambda \in \{S^{(m)}_+, S^{(m)}_-\}$

Dvodel\v cni \v casovni propagator za feromagnet se napi\v se kot
\[
	U^{(2)}_\text{fer}(t) = \e^{-it}\cos(2t)\mathbbm{1}_4 + i\e^{-it}\sin(2t) \begin{bmatrix}
		1 &  &  &  \\
		 & 0 & 1 &  \\
		 & 1 & 0 &  \\
		 &  &  & 1 \end{bmatrix},
\]

kar je v bistvu $U^{(2)}(\beta \to -it)$.

\end{document}
