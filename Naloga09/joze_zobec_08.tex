\documentclass[12pt, a4 paper]{article}
\usepackage[slovene]{babel}
\usepackage[T1]{fontenc}
\usepackage[utf8]{inputenc}
\usepackage{amsmath, amssymb, bbm, graphicx, float, caption}

\newcommand{\e}{
	\ensuremath{\mathrm{e}}
}

\newcommand{\ch}{
	\operatorname{ch}
}

\newcommand{\sh}{
	\operatorname{sh}
}

\renewcommand{\d}{
	\ensuremath{\mathrm{d}}
}

\newcommand{\rang}{
	\operatorname{rang}
}

\captionsetup{
	width=0.8\textwidth,
	labelfont=it
}

\begin{document}

\section{Implementacija postopka TEBD}

Valovna funkcija pri \v casu $t$ se zapi\v se lahko kot
\[
	|\psi(t)\rangle = \e^{-itH} |\psi(t = 0)\rangle
\]
\v Ce zelimo isto stanje dobiti pri neki inverzni temperaturi $\beta$, je pravilni pristop, da $\beta$ obravnavamo
kot negativni imaginarni \v cas, tj.
\[
	|\psi_{\beta}\rangle = |\psi(-i\beta)\rangle = \e^{(-i\beta)(-i H)}|\psi(t = 0)\rangle = \e^{-\beta H}
		|\psi(\beta=0)\rangle.
\]
Iz prej\v snje naloge se spomnimo, da lahko vse ugotovimo iz dvodel\v cnega hamiltoniana, $h^{(2)}_{1,2}$. V soglasju s tem
bomo definirali dvodel\v cni propagator:
\[
	U^{(2)}(\beta) = \exp\Big({-\beta h^{(2)}_{1,2}}\Big) = \exp\bigg(-\beta\sum_{\lambda \in \{x,y,z\}}\sigma^\lambda
		\otimes \sigma^\lambda\bigg).
\]
Ker matrike $\sigma^\lambda \otimes \sigma^\lambda$ med seboj komutirajo $\forall \lambda$, lahko izraz zapi\v semo kot
produkt treh matrik (vi\v sji \v cleni razvoja BCH so enaki ni\v c)
\begin{equation}
	U^{(2)}(\beta) \equiv \prod_{\lambda \in \{x,y,z\}} \exp\big(-\beta \sigma^\lambda \otimes \sigma^\lambda\big).
\end{equation}
Razvoj lepo konvergira in velja
\[
	\Big(\sigma^\lambda \otimes \sigma^\lambda\Big)^n = \left\{
		\begin{array}{cc}
			\mathbbm{1}; & n = 2m, m \in \mathbb{N} \\
			\sigma^\lambda \otimes \sigma^\lambda; & n = 2m - 1, m \in \mathbb{N}
		\end{array}
	\right.
\]
Ko to upo\v stevamo, ugotovimo, da je
\[
	U^{(2)}(\beta) = \e^{-\beta h^{(2)}_{1,2}} = \e^{\beta}\begin{bmatrix}
		\e^{-2\beta} & 0 & 0 & 0\\
		0 & \ch(2\beta) & -\sh(2\beta) & 0 \\
		0 & -\sh(2\beta) & \ch(2\beta) & 0 \\
		0 & 0 & 0 & e^{-2\beta}
	\end{bmatrix},
\]
tj. dvodel\v cni temperaturni propagator za anti-feromagnet. Dvodel\v cni \v casovni propagator za \emph{feromagnet} se
napi\v se kot
\[
	U^{(2)}_\text{fer}(t) = \e^{-it}\cos(2t)\mathbbm{1}_4 + i\e^{-it}\sin(2t) \begin{bmatrix}
		1 &  &  &  \\
		 & 0 & 1 &  \\
		 & 1 & 0 &  \\
		 &  &  & 1 \end{bmatrix},
\]
kar je v bistvu $U^{(2)}(\beta \to -it)$.

\subsection{Spinska korelacija}
Spinska korelacija je definirana kot
\[
	C_{jk} \sim \langle \sigma_j^z \sigma_k^z \rangle = \langle \psi| \sigma_j^z \sigma_k^z |\psi \rangle,
\]
kjer je o\v citno
\[
	\sigma_j^z \equiv \mathbbm{1}_{2^{j-1}} \otimes \sigma^z \otimes \mathbbm{1}_{2^{n-j}}.
\]
Na\v s na\v crt je, da bomo uporabili ena\v cbe (76), (77), (78) in (79) iz skripte. Torej potrebujemo matrike
$\tilde{\mathbf{A}}^{(j)}_{s_j}$, ki generirajo $\psi^*_{s_1,s_2,\ldots,s_n}$. Trdimo, da $\tilde{\mathbf{A}}^{(j)}_{s_j} =
\big(\mathbf{A}^{(j)}_{s_j}\big)^*$, kar lahko zlahka doka\v zemo:
\begin{align*}
	\psi_{s_1,s_2,\ldots,s_n} &= \mathbf{A}^{(1)}_{s_1} \mathbf{A}^{(2)}_{s_2} \cdots \mathbf{A}^{(n)}_{s_n} \\
	\psi^*_{s_1,s_2,\ldots,s_n} &= \Big(\mathbf{A}^{(1)}_{s_1} \cdots \mathbf{A}^{(n)}_{s_n}\Big)^*
\end{align*}
Na tem mestu bi sicer lahko rekli, da je $\psi^*_{s_1,s_2,\ldots,s_n}$ skalar in da je kompleksno konjugiranje skalarja
enakovredno hermitskemu konjugiranju (tj. hermitiranju) vendar tega ne smemo, saj bi nam to obrnilo vrstni red mno\v zenja!
Mno\v zimo namre\v c urejeno z "`leve proti desni"'. Tj. zgornji izraz enostavno zapi\v semo kot
\begin{align*}
	\psi^*_{s_1,s_2,\ldots,s_n} &= \big(\mathbf{A}^{(1)}_{s_1}\big)^* \big(\mathbf{A}^{(2}_{s_2}\big)^* \cdots
		\big(\mathbf{A}^{(n)}_{s_n}\big)^* \\
		&= \tilde{\mathbf{A}}^{(1)}_{s_1} \cdots \tilde{\mathbf{A}}^{(n)}_{s_n}.
\end{align*}
Ta produkt je \v ze urejen po spinih, vidimo da je res
\[
	\tilde{\mathbf{A}}^{(j)}_{s_j} \equiv \big(\mathbf{A}^{(j)}_{s_j}\big)^* = \mathbf{A}^{(j)}_{s_j}{}^*
\]
Na\v s operator $O_{s,s'}$ je kar $\sigma^z$. Iz njega moramo dobiti operatorje $\mathbf{V}^{(j)}$ iz ena\v cbe (77). Dobiti
jih je precej enostavno. Imamo vsoto po \v stirih elementih. Od tega sta dva enaka ni\v c, saj $\sigma^z_{\uparrow\downarrow}
= \sigma^z_{\downarrow\uparrow} = 0$. Torej
\begin{align}
	\mathbf{V}^{(j)} &= \sigma^z_{\downarrow\downarrow} \mathbf{A}^{(j)}_{\downarrow}{}^* \otimes
		\mathbf{A}^{(j)}_{\downarrow} + \sigma^z_{\uparrow\uparrow} \mathbf{A}^{(j)}_{\uparrow}{}^*
		\otimes \mathbf{A}^{(j)}_{\uparrow} = \notag \\
		&= \mathbf{A}^{(j)}_{\downarrow}{}^* \otimes \mathbf{A}^{(j)}_{\downarrow} - \mathbf{A}^{(j)}_\uparrow{}^*
		\otimes \mathbf{A}^{(j)}_\uparrow.
\end{align}
Analogno lahko vidimo, da $\mathbf{L} \equiv \mathbf{T}^{(1)}$ in $\mathbf{R} \equiv \mathbf{T}^{(n)}$ in
\begin{align}
	\mathbf{T}^{(j)} = \mathbf{A}^{(j)}_\downarrow{}^* \otimes \mathbf{A}^{(j)}_\downarrow +
		\mathbf{A}^{(j)}_\uparrow{}^* \otimes \mathbf{A}^{(j)}_\uparrow .
\end{align}
Torej je $\langle \sigma^z_j \sigma^z_k \rangle$
\[
	\langle \psi |\sigma^z_j \sigma^z_k | \psi\rangle = \mathbf{T}^{(1)} \cdots \mathbf{T}^{(j-1)} \mathbf{V}^{(j)}
		\mathbf{T}^{(j+1)} \cdots \mathbf{T}^{(k-1)} \mathbf{V}^{(k)} \mathbf{T}^{(k+1)}\cdots \mathbf{T}^{(n)},
\]
pri \v cemer moramo biti na robne pogoje, tj. $2 \leq j < k \leq n-1$. Nimamo periodi\v cnih robnih pogojev, torej nimamo
prave translacijske simetrije tudi za naklju\v cno \v zrebane vektorje. Vendar pa moramo ta izraz izpovpre\v citi po
nekem \v casu, tj.
\[
	C_{ij}(\beta) \equiv \lim_{t \to \infty} \frac{1}{t}\int_0^t \d \tau \langle \psi_\beta(\tau)|\sigma_j^z\sigma_k^z
		|\psi_\beta(\tau)\rangle
\]
\v Ce bomo za\v cetno stanje dovolj ohladili ($\beta \to \infty$) bomo slej, ko prej dosegli osnovno stanje (fizikalna
interpretacija je o\v citna, matemati\v cno pa je tudi res, saj je po absolutni vrednosti za na\v s hamiltonian vedno
najve\v cja lastna vrednost osnovnega stanja, kar pomeni, da v bistvu i\v s\v cemo osnovno stanje s poten\v cno metodo).
Korelacijska funkcija osnovnega stanja je potem kar
\[
	C_{ij}^\infty = C_{ij}(\beta \to \infty) = \lim_{t \to \infty} \frac{1}{t} \int_0^t \d\tau \lim_{\beta \to \infty}
		\langle\psi_\beta(\tau)|\sigma_j^z\sigma_k^z|\psi_\beta(\tau)\rangle.
\]
Vrstni red limitiranja je tu pomemben. Najprej moramo limitirati $\beta$, saj moramo na\v se naklju\v cno stanje najprej
ohladiti in \v sele potem povpre\v citi po \v casu. Stanje je treba sproti \v se normirati, tj. za splo\v sni primer se izraz
glasi
\begin{equation}
	C_{ij}^\infty = C_{ij}(\beta \to \infty) = \lim_{t \to \infty}\frac{1}{t}\int_0^t \d\tau \lim_{\beta \to \infty}
		\frac{\langle\psi_\beta(\tau)|\sigma^z_j\sigma^z_k|\psi_\beta(\tau)\rangle}
		{\langle\psi_\beta(\tau)|\psi_\beta(\tau)\rangle}
\end{equation}
Vendar, pa je osnovno stanje lastno stanje in se ne spreminja s \v casom, torej ni treba povpre\v citi po \v casu.

\subsection{Domenska stena}
Na\v se stanje $|\psi\rangle$ lahko predstavimo kot vektor $\psi$ v prostoru spinskih konfiguracij. Te lahko o\v stevil\v cimo
tako, da npr. spin `$\uparrow$' na mestu $n$ predstavlja $1$ na mestu $n$ in z $0$ bi na istem mestu ozna\v cili spin
`$\downarrow$'. Vidimo, da dobimo binarno \v stevilo, ki ga lahko pretvorimo v deseti\v skega, kjer na\v si indeksi te\v cejo
od $0$ do $2^n-1$ (veriga ima $n$ spinov, tj. $2^n$ mo\v znih konfuguracij).

V na\v sem za\v cetnem stanju je polovico spinov `$\uparrow$', polovica pa jih je `$\downarrow$' (ali obratno). V prostoru
konfiguracij bomo ozna\v cili to konfiguracijo z deseti\v skim \v stevilom `$k$'. To \v stevilo lahko za verigo dol\v zine
$n = 2m$ izra\v cunamo vnaprej in sicer
\begin{multline}
	k_+^{(m)} = \overbrace{11\ldots1}^{m}\overbrace{00\ldots0}^{m}{}_2 = 2^{2m-1} + 2^{m-2} + \ldots + 2^m = \\
		2^m(1 + 2 + 4 + \ldots + 2^{m-1}) = 2^m (2^m - 1).
\end{multline}
Zgornji izraz v oklepaju bi potemtakem ustrezal
\begin{equation}
	k_-^{(m)} = \overbrace{00\ldots0}^{m}\overbrace{11\ldots1}^{m}{}_2 = 1 + 2 + \ldots + 2^{m-1} = 2^m-1.
\end{equation}
Na\v se za\v cetno stanje, $k$, je torej bodisi enotski vektor v smeri $k_+$, bodisi v smeri $k_-$. Tukaj privzeli, da
indeksi te\v cejo od $0$ do $n-1$. \v Ce na\v si indeksi te\v cejo od $1$ do $n$, moramo vzeti
$k \in \{k_+ + 1, k_- + 1\}$. V algoritmu sem vzel $k \equiv k_-$, tj. prvih $m$ spinov ka\v ze gor (\v stejemo jih iz
desne proti levi, ker skrajno desni bit predstavlja enice v binarnem zapisu).

Moj MPA postopek je nekoliko nerodno ra\v cunal matriko $\big[\lambda^{(j)}\big]^{-1}$ in sicer tako, da je vzel diagonalo
in jo obrnil. \v Ce je na diagonali kak\v sno \v stevilo `$0$', potem inverz ni bil definiran, saj ne smemo deliti z `$0$'. Za
re\v sevanje domenske stene, ki ima na za\v cetku samo eno neni\v celno singularno vrednost to predstavlja resen problem.
Lahko bodisi popravimo izra\v cun inverza (kar je dokaj trivialno), lahko pa za vajo ra\v cunamo s polnimi vektorji in
preverimo, da na\v s postopek res deluje tudi v teh na\v cinih "`po ovinkih"'. Izbral sem postopek "`po ovinkih"'.

Moj za\v cetni vektor, $\psi$, sem razvil po bazi spinskih konfiguracij,
\[
	\psi = \sum_{j=1}^n\frac{\alpha_j}{|\alpha_k|}\delta_{jk}\hat{e}_j,
\]
kjer so $\alpha_j \in \mathbb{C}$ izbrane po Gaussovi porazdelitvi okrog ni\v cle, in velja $\psi^\dagger \psi = 1$.
Da se ognemo prej omenjenim nev\v se\v cnostim, definiramo na\v se stanje kot
\[
	\psi = \phi_1 - \phi_2 = \underbrace{\sum_{j = 1}^n\hat{e}_j \frac{\alpha_j}{|\alpha_k|}(1 + \delta_{jk})}_{\phi_1}
		- \underbrace{\sum_{j=1}^n\hat{e}_j \frac{\alpha_j}{|\alpha_k|}}_{\phi_2}.
\]
Tako $\phi_1$ kot $\phi_2$ sta polna vektorja, zato ne bomo imeli problemov z MPA. Spinski profil verige dobimo enostavno
kot
\[
	\langle\psi|\sigma^z_j|\psi\rangle = \langle\phi_1 - \phi_2|\sigma^z_j|\phi_1 - \phi_2\rangle =
	\langle\phi_1|\sigma^z_j|\phi_1\rangle + \langle\phi_2|\sigma_j^z|\phi_2\rangle - 2\mathrm{Re}\left\{
	\langle\phi_1|\sigma^z_j|\phi_2\rangle\right\},
\]
ki jih lahko u\v cinkovito ra\v cunamo z matrikami $\mathbf{T}^{(j)}$ in $\mathbf{V}^{(j)}$. Na\v se kombinirano stanje
$\psi$ je \v se vedno enotski vektor v smeri $k$, vendar ga ra\v cunamo z dvema polnima vektorjema.
\v Ce algoritem res deluje pravilno, moramo tudi v tem primeru dobiti pravi rezultat.

\subsection{Napaka TEBD algoritma}

Glavna mo\v c TEBD algoritma je ta, da s pomo\v cjo singularnega razcepa odstranimo majhne Schmidtove koeficiente in
prihranimo pri ra\v cunskem \v casu in pri ra\v cunalni\v skem spominu. Napaka pride tudi od izbire na\v sega integratorja.
Izbrali si bomo simplekti\v cnega, ki pa \v se vedno nosi napako.

\subsubsection{Rezanje matrik}

Rezanje matrik pomeni, da za vsako biparticijo sprejmemo kve\v cjemu $M$ Shmidtovih koeficientov. To pomeni, da
\[
	\rang\lambda^{(j)} \leq M,\ \forall j.
\]
Re\v zemo tako, da v primeru ko se za nek $j \leq n-1$ zgodi  $\rang\lambda^{(j)} > M$, sprejmemo le prvih $M$ Schmidtovih
koeficientov in ostale zavr\v zemo. Temu primerno potem na istem mestu $j$ popravimo $B^{(j)}_{s_j}$, tako da vzamemo le
prvih $M$ stolpcev in popravimo \v se $B^{(j+1)}_{s_{j+1}}$, tako da sprejmemo le prvih $M$ vrstic. Pri tem napravimo majhno
napako (proporcionalna je vsoti kvadratov zavr\v zenih Schmidtovih koeficientov).

Pomembno je kdaj re\v zemo. Ugotovil sem, da dobimo najbolj\v se rezultate, \v ce re\v zemo sproti ko izvajamo
Trotter-Suzukijevo shemo, za\v cetni MPA pa pustimo tak, kot je. \v Ce namre\v c prire\v zemo \v ze za\v cetku, bomo dobili
pribli\v zno pol ni\v zjo energijo.

Re\v zemo lahko tako, da $2^m > M > 1$, za verigo dol\v zine $n = 2m$.

\subsubsection{Trotter-Suzuki}

Za propagacijo uporabimo simplekti\v cni propagator Trotter-Suzuki. Za \v casovno propagacijo (domenska stena) uporabimo kar
shemo `$S_2$' iz drugega poglavja. Za temperaturno propagacijo pa imamo disipativni problem (norma vektorja se burno
spreminja), zaradi tega bomo na tem mestu uporabili shemo `$S_3$' s kompleksnimi koeficienti.

\section{Rezultati}

Uporabil sem programski paket \texttt{Octave} zaradi tega, ker sem TEBD implementiral relativno hitro in pravilno (nisem
se izgubljal okrog indeksov, programski jezik \texttt{C} je glede tega dosti bolj nepregleden.

Za potrebe temperaturne propagacije sem stanje ohlajal do $\beta_\text{max} = 10$ v 500 korakih.

\subsection{Osnovno stanje}

Pri\v cnemo z naklju\v cnim kompleksnim Gaussovim vektorjem. Dobimo stolpec $\ln\mathcal{N}(\beta_i)$ in $\beta_i$, kjer
$\mathcal{N}(\beta_i) \equiv \sqrt{\langle\psi_{\beta_i}|\psi_{\beta_i}\rangle}$. Veljati mora
\[
	-E_0\beta_i + \delta = \ln\mathcal{N}(\beta_i),
\]
tj. na koncu $E_0$ in $\delta$ dobimo npr. z linearno regresijo oz. v mojem primeru z minimizacijo napake prek
Marqart-Levenbergovega algoritma, ki da bolj\v se rezultate. Minimizacijo sem izvajal prek programa \texttt{gnuplot}, ki
ga uporabljam tudi sicer za izris grafov. Slika~\ref{energija2} prikazuje primer kako se spreminja `$-\ln\mathcal{N(\beta)}$' z
inverzno temperaturo $\beta$. Prikazal sem le en graf, saj so ostali isti, le smerni koeficient premice je druga\v cen.

\begin{figure}[H]\centering
	\input{energija-2-0.tex}
	\caption{Ohlajanje verige dol\v zine $4$ brez rezanja matrik.}
	\label{energija2}
\end{figure}

\begin{table}[H]\centering
	\caption{Ta tabela prikazuje dobljeno energijo osnovnega stanja za razli\v cne izbire rezanja, $M$
		in za razli\v cne dol\v zine verige. Za primerjavo je poleg tudi logaritem odstopanja od vrednosti,
		dobljene diagonalizacijo matrike. $M = 0$ pomeni, da matrik nismo prirezali. Vidimo, da odre\v zemo
		polovico Schmidtovih koeficientov in \v se vedno dobimo presenetljivo dobre rezultate.}
	\vspace{6pt}
	\begin{tabular}{c|c|c|c}
		$n$ & $M$ & $E_0$ & $\log_{10}\big|E_0 - E_0^\text{diag}\big|$ \\
		\hline
		$4$ & $0\ (4)$ & $-6.4641$ & $-5.79$ \\
		    & $2$ & $-6.1582$ & $-0.51$ \\
		\hline
		$6$ & $0\ (8)$ & $-9.4743$ & $-5.81$ \\
		    & $7$ & $-9.4723$ & $-2.97$ \\
		    & $6$ & $-9.9720$ & $-0.60$ \\
		    & $2$ & $-9.2664$ & $-0.15$ \\
		\hline
		$8$ & $0\ (16)$ & $-13.4997$ & $-4.52$ \\
		    & $15$ & $-13.4997$ & $-4.52$ \\
		    & $11$ & $-13.4997$ & $-4.52$ \\ 
		    & $8$ & $-13.4990$ & $-3.14$ \\
		    & $4$ & $-13.4311$ & $-1.16$ \\
		    & $2$ & $-12.3790$ & $0.05$ \\
		\hline
		$10$ & $0\ (32)$ & $-17.0321$ & $-4.39$ \\
		     & $4$ & $-16.8549$ & $-0.75$ \\
		     & $8$ & $-16.9405$ & $-1.04$ \\
		     & $12$ & $-17.0183$ & $-1.86$ \\
		     & $16$ & $-17.0321$ & $-4.39$ \\
		\hline
		$12$ & $0\ (64)$ & $-20.5684$ & $-4.43$ \\
		     & $4$ & $-20.2323$ & $-0.47$ \\
		     & $8$ & $-20.2677$ & $-0.52$ \\
		     & $12$ & $-20.4912$ & $-1.11$ \\
		     & $16$ & $-20.5150$ & $-1.27$ \\
		     & $24$ & $-20.5651$ & $-2.49$ \\
		     & $32$ & $-20.5684$ & $-4.43$
	\end{tabular}
	\label{tab1}
\end{table}

\subsection{Spinska korelacijska matrika}

V \v casovno propagacijo. Korelacijska matrika za osnovno stanje verige dol\v zine $4$ je kar
\begin{equation}
	C = \begin{bmatrix} 1 & 1/3 \\ 1/3 & 1\end{bmatrix}
\end{equation}
Ostale korelacijske matrike so ve\v cje (za verigo $n = 10$ in $n = 12$) na grafih~\ref{kor1},~\ref{kor2} in~\ref{kor3}.

\begin{figure}[H]\centering
	\input{korelacija-10-0.tex}
	\vspace{-32pt}
	\caption{Spin-spin korelacijska matrika za verigo dol\v zine 10, brez rezanja. Opazimo izrazito blo\v cno
		obliko, kar sicer ni ni\v c nenavadnega.}
	\label{kor1}
\end{figure}

\begin{figure}[H]\centering
	\input{korelacija-12-0.tex}
	\vspace{-32pt}
	\caption{Spin-spin korelacijska matrika za verigo dol\v zine 12, brez rezanja. Opazimo zelo podobno sliko, kot
		prej. Spet imamo izrazito blo\v cnatost slike. Po barvi vzorca razberemo, da so vrednosti ob diagonali
		podobne.}
	\label{kor2}
\end{figure}

\begin{figure}[H]\centering
	\input{korelacija-12-8.tex}
	\vspace{-32pt}
	\caption{Spin-spin korelacijska matrika za verigo dol\v zine 12, pri \v cemer smo za vsak razcep obdr\v zali
		najve\v c $M = 8$ Schmidtovih koeficientov. Imeli smo natan\v cnost le na eno decimalno mesto, zaradi
		\v cesar je korelacijska matrika popa\v cena.}
	\label{kor3}
\end{figure}

\subsection{Domenska stena}

Domensko steno sem ra\v cunal za spinsko verigo z 12 spini. Matrike sem sprva rezal tako, da sem zavrgel polovico, tj.
obdr\v zal najve\v c 32 Schmidtovih koeficientov, saj to sode\v c po tabeli~\ref{tab1} zado\v s\v ca. Temu \v zal ni
tako -- za propagiranje sem namre\v c uporabljal razli\v cni simplekti\v cni shemi in izka\v ze se, da $S_2$ ni dovolj
dobra, kar se vidi na sliki~\ref{fail}. Odlo\v cil sem se, da bom obdr\v zal kar vse koeficiente. Rezultat (ki je pravi)
je na sliki~\ref{bravo}

\begin{figure}[H]\centering
	\input{profil-12-32.tex}
	\vspace{-32pt}
	\caption{Kot vidimo, se vsak spin propagira \v cisto po svoje, ne vidimo nobene simetrije. Vidimo tudi, da so
		amplitude \v cedalje ve\v cje -- spini so sprva bodisi $-1$, bodisi $1$, potem pa postanejo amplitude
		ogromne (med $100$ in $-150$.}
	\label{fail}
\end{figure}

\begin{figure}[H]\centering
	\input{profil-12-0.tex}
	\vspace{-32pt}
	\caption{Graf je videti dosti bolj organiziran. Takoj opazimo (anti-)simetrijo med levo in desno stranjo.}
	\label{bravo}
\end{figure}

Zdaj, ko smo dobili nekak\v sen "`tloris"' problema, poglejmo \v se, projekcijo na \v casovno os, kar ka\v ze
slika~\ref{spini}.

\begin{figure}[H]\centering
	\input{spini.tex}
	\caption{\v Ceprav je graf~\ref{bravo} kazal harmonijo, je ta graf na prvi pogled ne ka\v ze. Imamo namre\v c
		kar veliko vzbujenih frekvenc. Zanimivo bi bilo videti frekven\v cni spekter za posamezen spin, a
		pustimo tole za kdaj drugi\v c.}
	\label{spini}
\end{figure}

\end{document}
