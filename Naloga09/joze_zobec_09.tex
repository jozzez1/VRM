\documentclass[12pt, a4 paper]{article}
\usepackage[slovene]{babel}
\usepackage[T1]{fontenc}
\usepackage[utf8]{inputenc}
\usepackage{amsmath, amssymb, bbm, graphicx, float, caption}

\captionsetup{
	width=0.7\textwidth,
	labelfont=it,
	size=small
}

\renewcommand{\d}{
	\ensuremath{\mathrm{d}}
}

\newcommand{\e}{
	\ensuremath{\mathrm{e}}
}

\newcommand{\err}{
	\operatorname{err}
}

\newcommand{\V}{
	\ensuremath{V^\text{eff}}
}

\begin{document}

\begin{center}
\textsc{Vi\v sje ra\v cunske metode}\\
\textsc{2012/13}\\[0.5cm]
\textbf{9. naloga -- algoritem DFT}
\end{center}
\begin{flushright}
\textbf{Jo\v ze Zobec}
\end{flushright}

\section{Implementacija DFT}

Najprej potrebujemo dva integratorja -- prvi bo re\v seval Schroedingerjevo ena\v cbo za nek potencial $V(r)$, drugi
pa bo re\v seval Poissonovo ena\v cbo.

\subsection{Schroedingerjeva ena\v cba.}

Schroedingerjeva ena\v cba, ki jo bomo re\v sevali je
\[
	\bigg[-\frac{1}{2}\frac{\d^2}{\d r^2} + V(r) + \frac{\ell(\ell+1)}{r^2} - E\bigg]u(r) = 0,
\]
kjer je $u(r) = r\psi(r)$ in $\ell \equiv 0$ ($\psi$ je valovna funkcija, ki je za $\ell = 0$ radialno simetri\v cna, tj.
$\psi = \psi(r)$). \v Ce jo napi\v semo nekoliko druga\v ce, dobimo
\begin{equation}
	\bigg[\frac{\d^2}{\d r^2} + 2\Big(E - V(r)\Big)\bigg]u(r) = 0.
\end{equation}
Osnovno stanje Schr\" odingerjeve ena\v cbe bomo iskali s pomo\v cjo metode Numerova. Potrebujemo $u(r=0) = u_1$ in
$u (r = h) = u_2$. Ker je $u(r) = \psi/r$ vemo, da je $u_1 = 0$, $u_2$ pa je arbitraren (oz. dolo\v cimo ga iz
normalizacijskega pogoja). Energijo osnovnega stanja, $E$, dolo\v cimo iz robnega pogoja $u(r_\text{max}) = u_N = 0$. Ta
pogoj je namre\v c izpolnjen samo, ko je $E \approx E_\text{to\v cna}$.

Ne poznamo potenciala v ni\v cli: zaradi tega bomo namesto $u_1 = u(0)$ rekli $u_1 = u(r_\text{min}) = 0$ in enako tudi
za potencial. Manj\v si ko bo $r_\text{min}$, bolj natan\v cni bodo na\v si rezultati.

Ko imamo obliko osnovnega stanja za neko energijo $E$, jo lahko izra\v cunamo s pomo\v cjo strelske metode
tako, da $u(r_\text{max}) = 0$. Zelo hitro lahko pademo v podro\v cje sipalnih stanj (\v ce uporabljamo strelsko
metodo), zato bomo raje delali z bisekcijo, ki pa je bolj ra\v cunsko zahtevna. \v Ce namre\v c energijo i\v scemo v intervalu
dol\v zine $p$ in bi radi preciznost $10^{-\alpha}$ bomo morali napraviti $\log_2(p 10^\alpha)$ ra\v cunskih korakov, kar
je za velikostni red ve\v c kot \v ce bi bila funkcija dovolj pohlevna, in bi ni\v clo lahko iskali s sekantno metodo.
Slika~\ref{gr0} prikazuje funkcijo, katere ni\v clo moramo poiskati, to je $u_N(E)$.

\begin{figure}[H]\centering
	\input{energio.tex}
	\caption{Kot vidimo, se funkcija v bli\v zini $E = -0.5$ zelo strmo spreminja, poleg tega, da imamo tam v bli\v zini
		zelo strm ekstrem, zaradi \v cesar nas lahko zelo hitro vrz\v ce iz tira. Takega problema se je zato
		najbolje lotiti kar z bisekcijo in ne s sekantno metodo.}
	\label{gr0}
\end{figure}

Metoda Numerova ni bila edina s katero sem se lotil tega problema. Za poku\v sino sem posegel tudi po RK4 (Runge-Kutta reda 4),
kjer sem vmesne vrednosti potenciala aproksimiral z aritmeti\v cno sredino (tj. $V_{i + 1/2} = (V_i + V_{i+1})/2 +
\mathcal{O}(h^2)$) in tudi tako, da sem Laplace-ov operator diskretiziral do reda $\mathcal{O}(h^6)$ in iskal lastne
vrednosti. RK4 mi je dajal rezultate, ki so bili primerljivi z metodo numerova, iskanje z diagonalizacijo pa je dajalo
slab\v se rezultate: predvsem zato, ker je najmanj\v sa lastna vrednost bila tipi\v cno $\sim -10^4$, in \v ce bi hotel
ravno lastno vrednost v okolici $-0.5$, bi moral pose\v ci po celi diagonalizaciji matrike, kar pa ni bilo efektivno, sploh
glede na to, da bi za to moral opraviti $N^3$ operacij.

\subsection{Poissonova ena\v cba}

Integrator, ki re\v suje Poissonovo ena\v cbo bomo naredili tako, da bomo drugi odvod $U''(r_i)$
aproksimirali s kon\v cnimi diferencami
\[
	\frac{\d^2}{\d r^2} U(r)\Big|_{r = r_i} = \frac{U_{i+1} + U_{i-1} - 2U_i}{h^2},
\]
vendar je ta izraz natan\v cen le z napako $\mathcal{O}(h^2)$, metoda Numerova od prej, pa ima nata\v cnost
$\mathcal{O}(h^6)$. Uporabili bomo raje shemo s primerljivo nata\v cnostjo -- simetri\v cno diferen\v cno shemo reda
$\mathcal{O} (h^6)$, ki se glasi
\begin{equation}
	h^2 U''_i \approx \frac{1}{90}(U_{i+3} + U_{i-3}) - \frac{3}{20} (U_{i+2} + U_{i-2}) + \frac{3}{2}(U_{i+1}
		+ U_{i-1}) - \frac{49}{18}U_i
\end{equation}
V tem prepoznamo mno\v zenje vektorja $\underline{U}$ z matriko $L$, tj.
\begin{equation}
	\underline{U}'' \approx L\underline{U},\ L = \frac{1}{h^2}\begin{bmatrix}
		-49/18 & 3/2 & -3/20 & 1/90 & 0 & \ldots & 0 \\
		3/2 & \ddots & \ddots & \ddots & \ddots & \\
		-3/20 & \ddots & & & & \\
		1/90 & \ddots & & & & \\
		0 & \ddots & & & & \\
		\vdots & & & & & \\
		0 & & & & &
		\end{bmatrix},
\end{equation}
tj. pasata matrika, s tremi sub- in tremi super-diagonalami. Da dobimo $\underline{U}$ moramo re\v siti sistem
\begin{equation}
	L_{ij} U_j = y_i,\quad y_i = -u_i^2/r_i.
\end{equation}
Za re\v sevanje takega sistema obstajajo u\v cinkoviti algoritmi, ki pa jih ne bi omenjal. Za matriko dimenzije
$N\times N$ potrebujejo $\mathcal{O}(N)$ operacij, kar je bolj u\v cinkovito, kot \v ce bi re\v sevali npr.
\[
	U(r) = \int \frac{\d k}{2\pi} \int \d r' \frac{1}{k^2} \e^{-ik(r - r')} \frac{u^2(r')}{r'},
\]
ki ima ob uporabi FFT zahtevnost $\mathcal{O}(N\log N)$.

Kot pi\v se v skripti, moramo na koncu dodati \v se homogeno re\v sitev, da upo\v stevamo robni pogoj, $U(r) \to U(r) + kr$,
kjer je zelo pomembno, da je na\v se stanje pravilno normirano: $|u| = \sqrt{N/(r_\text{max} - r_\text{min})}$.

Sedaj imamo vse, da lahko napravimo LDA DFT z eno orbitalo.

\section{Rezultati}

Nalogo sem re\v sil z orodjem \texttt{Octave}. Parametri za DFT so $N = 12000$ to\v ck med $r_\text{min} = 10^{-20}$ in
$r_\text{max} = 15$, bisekcija se ustavi pri preciznosti $10^{-10}$, DFT pa se ustavi, ko je
\[
	\sum_{j = 1}^N |u_j^k - u_j^{k-1}|^2 + \sum_{j=1}^N |U_j^k - U_j^{k-1}|^2 < 10^{-8},
\]
tj. ko se tako $u$, kot $U$ bolj ali manj ne spreminjata ve\v c.

Iteracijo DFT pri\v cnemo z vodikovim atomom. To\v cna vrednost za lastno energijo je $-1/2$, jaz dobim $-0.499916$, kar
je solidno ujemanje. Funkcijo $u(r)$ in $U(r)$ vidimo na sliki~\ref{gr1}, napaka $U(r)$ glede na analiti\v cno re\v sitev je
prikazana na sliki~\ref{gr2}.

\begin{figure}[H]\centering
	\input{vodik.tex}
	\caption{Kot vidimo so robni pogoji lepo zado\v s\v ceni.}
	\label{gr1}
\end{figure}

\begin{figure}[H]\centering
	\input{uerr.tex}
	\caption{Natan\v nost ni tako, da bi se z njo hvalili -- nekako med tremi in \v stirimi decimalnimi mesti.
		$U_\text{ex}$ je ekstantna vrednost potenciala v $r$.}
	\label{gr2}
\end{figure}

Po iteraciji s postopkom LDA DFT dobimo re\v sitve za helijev atom: dobil sem $\epsilon = -0.51333$ in energijo osnovnega
stanja za helijev atom $E = -2.7177$. Na grafu~\ref{gr3} lahko vidimo na\v so novo funkcijo $u(r)$ in nov potencial $U(r)$.

\begin{figure}[H]\centering
	\input{helij.tex}
	\caption{Po iteracij smo dobili spremenjeno valovno funkcijo in spremenjen potencial, ki ustreza helijevemu
		atomu. Oblika je na mo\v c podobna, vendar ni ista, saj $u(r)$ in $U(r)$ pri\v cneta bolj strmo kot
		pri vodiku.}
	\label{gr3}
\end{figure}

\section{Zaklju\v cek}
Kot vidimo, smo z na\v so implementacijo DFT energijo $E$ dobili na dve decimalni mesti nata\v cno (tj. $E \approx -2.72$),
prav tako tudi energijo osnovnega stanja vodikovega atoma. Najbolj nata\v cno smo dobili $U(r)$ (eno decimalno mesto ve\v c),
najslab\v se pa je bilo ujemanje pri $\epsilon \approx -0.51$, kar je natan\v cno samo na eno decimalno mesto (prava vrednost
je $\epsilon = -0.52$). To bi lahko izbolj\v sal s pove\v cevanjem katere izmed omenjenih preciznosti in tudi z manj\v sanjem
koraka numerova oz. koraka pri matriki $L$. Vendar pa so rezultati zadosti solidni, kot zahteva naloga.

\end{document}
