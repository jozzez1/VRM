\documentclass[12pt, a4 paper]{article}
\usepackage[slovene]{babel}
\usepackage[T1]{fontenc}
\usepackage[utf8]{inputenc}
\usepackage{amsmath, amssymb, bbm, graphicx, float, caption}

\captionsetup{
	width=0.7\textwidth,
	labelfont=it,
	size=small
}

\renewcommand{\d}{
	\ensuremath{\mathrm{d}}
}

\newcommand{\e}{
	\ensuremath{\mathrm{e}}
}

\newcommand{\err}{
	\operatorname{err}
}

\newcommand{\V}{
	\ensuremath{V^\text{eff}}
}

\begin{document}

\begin{center}
\textsc{Vi\v sje ra\v cunske metode}\\
\textsc{2012/13}\\[0.5cm]
\textbf{9. naloga -- algoritem DFT}
\end{center}
\begin{flushright}
\textbf{Jo\v ze Zobec}
\end{flushright}

\section{Uvod}

Schroedingerjeva ena\v cba, ki jo bomo re\v sevali je
\[
	\bigg[-\frac{1}{2}\frac{\d^2}{\d r^2} + V(r) + \frac{\ell(\ell+1)}{r^2} - E\bigg]u(r) = 0,
\]
kjer je $u(r) = r\psi(r)$ in $\ell \equiv 0$ ($\psi$ je valovna funkcija, ki je za $\ell = 0$ radialno simetri\v cna, tj.
$\psi = \psi(r)$). \v Ce jo napi\v semo nekoliko druga\v ce, dobimo
\begin{equation}
	\bigg[\frac{\d^2}{\d r^2} + 2\Big(E - V(r)\Big)\bigg]u(r) = 0.
\end{equation}
Osnovno stanje Schr\" odingerjeve ena\v cbe bomo iskali s pomo\v cjo metode Numerova. Potrebujemo $u(r=0) = u_0$ in
$u (r = h) = u_1$, ki ga izra\v cunamo s pomo\v cjo razvoja v Taylorjevo vrsto z nastavkom
\[
	u(r) = \sum_{k = 1}^{\infty} c_k r^k.	
\]
Konstantni \v clen manjka, saj mora biti $\phi_1 = u(r)/r \in L^2$ normalizabilna. \v Ce uporabimo potencial 
$V(r) = -1/r$ in $u(r)$ razvijemo do petega reda (kot je tudi metoda Numerova -- ujemati se mora v
natan\v cnosti) dobimo rekurzijsko shemo
\[
	c_{k+1} = -\frac{2}{k(k+1)}(c_k + Ec_{k-1}),
\]
z za\v cetnimi pogoji $c_2 = -c_1$. Koeficienti so prikazani v tabeli~\ref{tab:koef}
\begin{table}[H]\centering
	\caption{Koeficienti za primer $V(r) = -1/r$ in $\ell = 0$ za Taylorjev razvoj $u(r)$ do reda
	$\mathcal{O}(h^6)$}
	\vspace{6pt}
	\begin{tabular}{c|l}
		$k$ & $c_k/c_1$ \\
		\hline
		$1$ & $1$ \\
		$2$ & $-1$ \\
		$3$ & $\frac{1}{3}(1 - E)$ \\
		$4$ & $\frac{1}{18}(4E - 1)$ \\
		$5$ & $-\frac{1}{30}(E^2 - \frac{1}{3}E - \frac{1}{6})$
	\end{tabular}
	\label{tab:koef}
\end{table}
S temi koeficienti lahko potem izra\v cunamo $u_0$ in $u_1$, ostale vrednosti pa prek metode Numerova. Vendar pa
je ta nabor koeficientov dober samo za potencial $V(r) = -1/r$. Za splo\v sen potencial, podan v obliki
$V(r_i) = V_i$ tega ne moremo narediti, razen \v ce tudi za potencial naredimo polinomsko interpolacijo prek
to\v ck, prav tako do reda $\mathcal{O}(h^6)$. Ker se bo potencial iterativno spreminjal, bodo koeficienti
$c_k$ \v ze takoj po prvi iteraciji postali "`slabi"'. Zaradi tega, bomo obliko osnovnega stanja izra\v cunali
kar z metodo Runge-Kutta reda 4 (v nadaljevanju RK4) z adaptivnim korakom. Koeficienti integratorja
$k_2$ in $k_3$ zahtevajo poznavanje $V_{i+1/2}$, ki ga dobimo z aproksimacijo $V_{i+1/2} =
(V_i + V_{i+1})/2 + \mathcal{O}(h^2)$, vendar nas to ne bo motilo.

Ko imamo obliko osnovnega stanja za neko energijo $E$, jo lahko izra\v cunamo s pomo\v cjo strelske metode
tako, da $u(r_\text{max}) = 0$. Zelo hitro lahko pademo v podro\v cje sipalnih stanj (\v ce uporabljamo strelsko
metodo), zato bomo raje delali z bisekcijo, ki pa je bolj ra\v cunsko zahtevna.

Integrator, ki re\v suje Poissonovo ena\v cbo bomo naredili tako, da bomo drugi odvod $U''(r_i)$
aproksimirali s kon\v cnimi diferencami
\[
	\frac{\d^2}{\d r^2} U(r)\Big|_{r = r_i} = \frac{U_{i+1} + U_{i-1} - 2U_i}{h^2},
\]
vendar je ta izraz natan\v cen le z napako $\mathcal{O}(h^2)$, RK4 od prej, pa je ima nata\v cnost
$\mathcal{O}(h^5)$. Uporabili bomo raje shemo s primerljivo nata\v cnostjo -- simetri\v cno shemo $\mathcal{O}
(h^6)$, ki se glasi
\begin{equation}
	h^2 U''_i \approx \frac{1}{90}(U_{i+3} + U_{i-3}) - \frac{3}{20} (U_{i+2} + U_{i-2}) + \frac{3}{2}(U_{i+1}
		+ U_{i-1}) - \frac{49}{18}U_i
\end{equation}
V tem prepoznamo mno\v zenje vektorja $\underline{U}$ z matriko $L$, tj.
\begin{equation}
	\underline{U}'' \approx L\underline{U},\ L = \frac{1}{h^2}\begin{bmatrix}
		-49/18 & 3/2 & -3/20 & 1/90 & 0 & \ldots & 0 \\
		3/2 & \ddots & \ddots & \ddots & \ddots & \\
		-3/20 & \ddots & & & & \\
		1/90 & \ddots & & & & \\
		0 & \ddots & & & & \\
		\vdots & & & & & \\
		0 & & & & &
		\end{bmatrix},
\end{equation}
tj. pasata matrika, s tremi sub- in tremi super-diagonalami. Da dobimo $\underline{U}$ moramo re\v siti
sistem
\begin{equation}
	L_{ij} U_j = y_i,\quad y_i = -u_i^2/r_i.
\end{equation}
Za re\v sevanje takega sistema obstajajo u\v cinkoviti algoritmi, ki pa jih ne bi omenjal. Za matriko dimenzije
$N\times N$ potrebujejo $\mathcal{O}(N)$ operacij, kar je bolj u\v cinkovito, kot \v ce bi re\v sevali npr.
\[
	U(r) = \int \frac{\d k}{2\pi} \int \d r' \frac{1}{k^2} \e^{-ik(r - r')} \frac{u^2(r')}{r'},
\]
ki ima ob uporabi FFT zahtevnost $\mathcal{O}(N\log N)$.

\end{document}
