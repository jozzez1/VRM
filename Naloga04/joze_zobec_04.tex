\documentclass[a4 paper, 12pt]{article}
\usepackage[slovene]{babel}
\usepackage[utf8]{inputenc}
\usepackage[T1]{fontenc}
\usepackage[small, width=0.8\textwidth]{caption}
\usepackage[pdftex]{graphicx}
\usepackage{amssymb, amsmath, mathtools, fullpage, float, hyperref, stackrel}

\begin{document}

\begin{center}
\textsc{Višje računske metode}\\
\textsc{2012/13}\\[0.5cm]
\textbf{4. naloga -- Kubitne verige}
\end{center}
\begin{flushright}
\textbf{Jože Zobec}
\end{flushright}

\newcommand{\e}{
	\ensuremath{\text{e}}
}

\section{Uvod}

Oznake in definicije bodo iste, kot so v skripti.

V tej nalogi imamo propagacijo po času, kot tudi po temperaturi (imaginarnem času).

Ker bomo imeli opravka z veliko množenji, matrika pa je le dimenzije $4 \times 4$ in je simetrična tridiagonalna
itd. se splača to transformacijo zakodirati naravnost v kodo. Če moramo transformirati vektor
$\underline{x} = (x_0, x_1, x_2, x_3)^T$, se ta po propagaciji s kompleksnim koeficientom $\alpha$ za kompleksni korak
$z$ transformira tako:

\begin{equation}
	\begin{bmatrix}
		x_0 \\
		x_1 \\
		x_2 \\
		x_3
	\end{bmatrix} \stackrel{U^{(2)}(\alpha z)}{\longrightarrow} \text{e}^{-\alpha z}
	\begin{bmatrix}
		\exp(2\alpha z) x_0 \\
		\cosh(2\alpha z)x_1 + \sinh(2\alpha z)x_2 \\
		\sinh(2\alpha z)x_1 + \cosh(2\alpha z)x_2 \\
		\exp(2\alpha z) x_3
	\end{bmatrix}.
	\label{eqn:U}
\end{equation}

Temperaturni propagator je $\sim \exp(-\beta H)$, in časovni je $\sim \exp (-itH)$, kjer sta $\beta$ in $t$ realni
količini\footnote{$t$ -- brezdimenzijski čas, $\beta$ -- recipročna brezdimenzijska temperatura}. Odtod vidimo, da
je $z$ čisto imaginarno število za čas in čisto realno število za temperaturni premik. Oz. drugače povedano

\[
	|\psi(t)\rangle = \e^{-itH}, \qquad |\psi(-i\beta)\rangle = \e^{-\beta H}.
\]

\section{Optimizacija propagatorja}

Naš propagator smo razbili na sode in na lihe. To pomeni da najprej po enačbi
\eqref{eqn:U} delujemo na stanja\footnote{delamo v bazi $\sigma_j^z$, $0$
pomeni spin "`dol"', $1$ pa pomeni spin "`gor"'}.

\begin{align*}
	&|s_N s_{N-1} \ldots s_3 \overbracket[0.5pt][5pt]{0\ 0}\rangle, \\
	&|s_N s_{N-1} \ldots s_3\ 0\ 1\rangle, \\
	&|s_N s_{N-1} \ldots s_3\ 1\ 0\rangle, \\
	&|s_N s_{N-1} \ldots s_3\ 1\ 1\rangle.
\end{align*}

Če je takih četvorčkov več, vedno delujemo na vsakega le enkrat. Potem se moramo premakniti na drug tak
blok četvorčkov, $s_4,s_3$:

\begin{align*}
	&|s_N s_{N-1} \ldots \overbracket[0.5pt][5pt]{0\ 0}s_2 s_1\rangle, \\
	&|s_N s_{N-1} \ldots 0\ 1\ s_2 s_1\rangle, \\
	&|s_N s_{N-1} \ldots 1\ 0\ s_2 s_1\rangle, \\
	&|s_N s_{N-1} \ldots 1\ 1\ s_2 s_1\rangle.
\end{align*}

Iskanje ustreznih četvorčkov za biti časovno zahteven posel in nam lahko oteži že tako časovno zahteven
izračun, saj ne vemo, kako so tile četvorčki urejeni v decimalnem zapisu. Izkaže se, da
lahko iz prvega bloka, četvorkčkov, ki so VEDNO urejeni že v decimalnem (tj. lahko
zapišemo njihove indekse kar kot \{{\tt [1, 2, 3, 4]}, {\tt [5, 6, 7, 8], \ldots}\}), pridemo v poljuben blok
z ustreznim zamikom bitov (seveda z ustreznimi periodičnimi pogoji, za katere poskrbi modulo po največjem možnem indeksu za dano dolžino verige). Če želimo
iz prvega sodega bloka priti na $n$-ti sodi blok, potem bite zamaknemo za $2n$ v levo. Če želimo iz sodega
priti v $n$-ti lihi blok, potem bite zamaknemo za $2n - 1$ v levo. Modulo že sam po sebi poskrbi za naše
periodične robne pogoje.

Zamikanje bitov je zelo hitra operacija, tako to delovanja našega programa močno zmanjša, prav tako pa nam
ni treba računati lihih in sodih posebej.

Zamik bitov z $n$ v levo števila $x$ je ekvivalentno $x \leftarrow x\cdot2^n$ (mod $x_\text{max}$), vendar je čisti
zamik hitrejši, tako da priporočam tega, če ga programsko okolje dopušča, saj
je ta že vgrajen v samo elektroniko naših procesorjev.

\section{Optimizacija hamiltoniana}

Stanja bom od sedaj številčil tako kot v programu, tj. namesto od $1\ldots N$,
raje od $0 \ldots N-1$.

Propagator smo dobili tako, da smo Hamiltonian razcepili na operatorja $A$ in $B$, eden delce "`prime"' na sodih,
drugi pa na lihih mestih. Če to matriko zapišemo zgolj za dvodelčno stanje, lahko hamiltonian zapišemo s pomočjo
matrike $A^{(2)}$, ki sedaj deluje le na enem kubitnem paru:

\[
	A^{(2)} = \begin{bmatrix}
		1 &  &  &  \\
		&-1 &\ 2 &  \\
		 & \quad 2 & -1 &  \\
		 &  &  & 1
		\end{bmatrix} = h_{01},
\]

kjer moramo s to matriko izmenično delovati na pare fermionov -- enkrat na vse sode, drugič na vse lihe. Hamiltonian
ima še dele, s katerim delujemo na lihe. Radi bi ga zapisali v isti bazi, kot $h_{01}$, da ga bomo lahko sešteli
in tako imeli le en Hamiltonski blok za dva kubita. Po izračunu dobimo

\[
	h_{10} =  \begin{bmatrix}
		1 &  &  &  \\
		&-1 &\ 2 &  \\
		 & \quad 2 & -1 &  \\
		 &  &  & 1
		\end{bmatrix} = h_{01}
\]

kar lahko z nekaj algebre potrdimo. Po definiciji mora veljati

\begin{equation}
	H^{(2)} = h_{01} + h_{10} = 2h_{01},
\end{equation}

ki ima lastne vrednosti $(-6, 2, 2, 2)$.
Če je naša numerična simulacija pravilna, moramo dobiti torej dobiti $\langle H^{(2)} \rangle_\beta
\stackrel{\beta \to \infty}{\longrightarrow} -6$. Daljše kubitne verige bodo osnovno skalirale proti 0, ker imamo
anti-feromagnet. Recimo, za verigo dolžine $3$ se lahko v potu svojega obraza prebijemo do hamiltoniana

\begin{equation}
	H^{(3)} = \begin{bmatrix}
	3 & 0 & 0 & 0 & 0 & 0 & 0 & 0 \\
	0 & -1 & 2 & 0 & 2 & 0 & 0 & 0 \\
	0 & 2 & -1 & 0 & 2 & 0 & 0 & 0 \\
	0 & 0 & 0 & -1 & 0 & 2 & 2 & 0 \\
	0 & 2 & 2 & 0 & -1 & 0 & 0 & 0 \\
	0 & 0 & 0 & 2 & 0 & -1 & 2 & 0 \\
	0 & 0 & 0 & 2 & 0 & 2 & -1 & 0 \\
	0 & 0 & 0 & 0 & 0 & 0 & 0 & 3
	\end{bmatrix},
\end{equation}

ki se v diagonalni obliki zapiše kot $\text{diag}(-3, -3, -3, -3, 3, 3, 3, 3)$.

Vendar pa je narediti tako matriko težko in se potrati veliko procesorskega časa s Kroneckerjevim produktom,
nato pa še z matričnim množenjem. Vendar pa mi lahko isto naredimo na boljši način. Izkoristimo namreč lahko
to, da je naš vektor $\underline{x}$ v resnici nekakšen spinor

\[
	\underline{x} = x_0|000\rangle + x_1|001\rangle + x_2|010\rangle + x_3|011\rangle + \ldots
\]

komponente pa ima prikladno urejene. Naš hamiltonian je simetričen preko diagonale $H_{00} - H_{NN}$,
vendar pa je simetričen tudi preko diagonale $H_{0N} - H_{N0}$, kar pomeni, da lahko število ničel in
enic v stanju zamenjamo in bo matrični element isti.

Kakorkoli že, če naše indekse prepišemo v binarno, opazimo, da je hamiltonian odvisen le od indeksa našega
stanja, saj denimo $|000\ldots0\rangle$ vedno stoji na prvem mestu, ne glede na dimenzijo. Prav tako
stanje $|000\ldots01\rangle$ ipd. Hamiltonska matrika je razpršena in za $N = 20$ je ogromna ter dejansko
lahko zasede gigabajte našega pomnilnika, s selektivnim postopkom pretvorbe v binarno, pa lahko v nekaj
sekundah izluščimo neničelne elemente in potem množimo le z njimi in tako prihranimo na času.

\section{Rezultati}

Dober pokazatelj, če naš integrator deluje pravilno, so grafi z energijo. Rezultati so konsistentni s pričakovanji.

\begin{figure}[H]
   \begin{center}
      \input{FreeEn.tex}
   \end{center}
   \vspace{-20pt}
   \caption{Prosta energija se pogljablja z večanjem $N$. Padec ni čisto potenčen, vendar sploh ni eksponenten.
	Odvodi na začetku so napačni, vendar številka ni -- v ničli namreč še vedno dobimo pol, ki ga program vrne
	pravilno. Če bi vzel krajši časovni korak, bi se to verjetno popravilo in rezultat ne bi bil dvomljiv
	zaradi odvoda.}
   \label{fig:pic1}
   \vspace{-10pt}
\end{figure}

Iz slike~\ref{fig:pic1} sklepamo, da bo v termodinamski limiti prosta energija
$F(\beta) \stackrel{\beta\to\infty}\longrightarrow - \infty$. Iz teh slik morda
to ni tako očitno, vendar prosta energija tu limitira proti končni vrednosti
za velike $\beta$.

\begin{figure}[H]
   \begin{center}
      \input{Energy.tex}
   \end{center}
   \vspace{-20pt}
   \caption{Energije osnovnih stanj za različne dolžine verig. Za $N = 2$ in $N = 4$ sem rešitev analitično primerjal
	in pride prava vrednost, tj. $-6$ in $-8$. Za šalo sem poizkusil tudi s prej omenjeno verigo dolžine 3, katero
	moramo obravnavati kot aperiodično, vendar tudi v tem primeru dobimo pravo energijo, to je $-3$.}
   \label{fig:pic2}
\end{figure}

Iz slike~\ref{fig:pic2} lahko sklepamo, da bo
$\langle H\rangle_\beta \stackrel{\beta\to\infty}{\longrightarrow} -\infty$. To
je anti-feromagnet, kar pomeni, da se bolj splača imeti čim bolj neurejeno
kubitno verigo. Ker imamo v termodinamski limiti neskončno stanj, pomeni da
moramo pričakovati tako energijo ter da je v bistvu tako obnašanje pričakovano.
Lihe verige, kot prej omenjeni hamiltonian za verigo dolžine tri se obnašajo
nekako čudno, vendar nisem šel preverjati energije takih verig.

\pagebreak

\subsection{Časovna korelacija}

Avtokorelacijo prvega spina smo definirali kot $C(t) = \langle \sigma^z_1(t) \sigma^z_1 (0) \rangle$, delamo v limiti
$\beta \to 0$, torej pri neskončni temperaturi, kar pomeni, da je imaginarna komponenta časa enaka 0.

\begin{figure}[H]
   \begin{center}
      \input{C-G5-N20.tex}
   \end{center}
   \vspace{-20pt}
   \caption{Korelacijska funkcija za 5 naključnih vektorjev. Naredil sem tudi z manjšo, vendar ni bilo razlike
	zaradi česar sklepam, da že zelo malo stanj dobro opiše naš problem.}
   \label{fig:pic}
   \vspace{-10pt}
\end{figure}

\begin{figure}[H]
   \begin{center}
      \input{C-G10-N16.tex}
   \end{center}
   \vspace{-20pt}
   \caption{Korelacijska funkcija za kubitno verigo dolžine 16, z 10 naključnimi vektorji. Graf je identičen
	tistemu, ki bi ga dobili s petimi in celo tremi naključnimi vektorji.}
   \label{fig:pic}
   \vspace{-10pt}
\end{figure}

\begin{figure}[H]
   \begin{center}
      \input{C-G5-N12.tex}
   \end{center}
   \vspace{-20pt}
   \caption{Slika je spet praktično enaka, zaključek je malo dvignjen pri 
	$t = 2$, kar nam namiguje, da se bližamo meji izven režima
	termodinamske limite.}
   \label{fig:pic}
   \vspace{-10pt}
\end{figure}

\begin{figure}[H]
   \begin{center}
      \input{C-G5-N10.tex}
   \end{center}
   \vspace{-20pt}
   \caption{Pred časom $t = 2$, ki je bil prej meja grafov, se korelacijska
	funkcija prične dvigovati, kar pomeni da je čas za večjo sliko. Vidimo,
	da imamo še vedno normalno avtokorelacijsko funkcijo.}
   \label{fig:pic}
   \vspace{-10pt}
\end{figure}

\begin{figure}[H]
   \begin{center}
      \input{C-G5-N8.tex}
   \end{center}
   \vspace{-20pt}
   \caption{Graf je tu očitno drugačen od tistega, ko je $N = 10$.}
   \label{fig:pic}
   \vspace{-10pt}
\end{figure}

\begin{figure}[H]
   \begin{center}
      \input{C-G5-N6.tex}
   \end{center}
   \vspace{-20pt}
   \caption{V tem primeru nimamo več dobre avtokorelacijske funkcije, saj
	je naša končna veriga prekratka, da bi dobro simulirala neskončno.
	Avtokorelacijska funkcija postane periodična.}
   \label{fig:pic}
   \vspace{-10pt}
\end{figure}

\begin{figure}[H]
   \begin{center}
      \input{C-G5-N4.tex}
   \end{center}
   \vspace{-20pt}
   \caption{Periodičnost je tu še bolj očitna.}
   \label{fig:pic}
   \vspace{-10pt}
\end{figure}

\begin{figure}[H]
   \begin{center}
      \input{C-G5-N2.tex}
   \end{center}
   \vspace{-20pt}
   \caption{Veriga dolžine 2 vsekakor ne more simulirati termodinamske limite. Kar dobimo
	tu je za konstanto premaknjena čista kosinusna funkcija.}
   \label{fig:pic}
   \vspace{-10pt}
\end{figure}

Iz teh slik smo se naučili, da za smiseln statistični opis potrebujemo kubitno
verigo zadostne dolžine (tj. za naš primer vsaj $N = 8$). Kako
hitro korelacijska funkcija pada proti nič lahko določimo s fitanjem.

\end{document}

