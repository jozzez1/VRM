\documentclass[a4 paper, 12pt]{article}
\usepackage[slovene]{babel}
\usepackage[utf8]{inputenc}
\usepackage[T1]{fontenc}
\usepackage[small, width=0.8\textwidth]{caption}
\usepackage[pdftex]{graphicx}
\usepackage{amssymb, amsmath, fullpage, float, hyperref, stackrel}

\begin{document}

\begin{center}
\textsc{Višje računske metode}\\
\textsc{2012/13}\\[0.5cm]
\textbf{4. naloga -- Kubitne verige}
\end{center}
\begin{flushright}
\textbf{Jože Zobec}
\end{flushright}

\newcommand{\e}{
	\ensuremath{\text{e}}
}

\section{Uvod}

Oznake in definicije bodo iste, kot so v skripti.

V tej nalogi imamo propagacijo po času, kot tudi po temperaturi (imaginarnem času).

Ker bomo imeli opravka z veliko množenji, matrika pa je le dimenzije $4 \times 4$ in je simetrična tridiagonalna
itd. se splača to transformacijo zakodirati naravnost v kodo. Če moramo transformirati vektor
$\underline{x} = (x_0, x_1, x_2, x_3)^T$, se ta po propagaciji s kompleksnim koeficientom $\alpha$ za kompleksni korak
$z$ transformira tako:

\begin{equation}
	\begin{bmatrix}
		x_0 \\
		x_1 \\
		x_2 \\
		x_3
	\end{bmatrix} \stackrel{U^{(2)}(\alpha z)}{\longrightarrow} \text{e}^{-\alpha z}
	\begin{bmatrix}
		\exp(2\alpha z) x_0 \\
		\cosh(2\alpha z)x_1 + \sinh(2\alpha z)x_2 \\
		\sinh(2\alpha z)x_1 + \cosh(2\alpha z)x_2 \\
		\exp(2\alpha z) x_3
	\end{bmatrix}.
	\label{eqn:U}
\end{equation}

Temperaturni propagator je $\sim \exp(-\beta H)$, in časovni je $\sim \exp (-itH)$, kjer sta $\beta$ in $t$ realni
količini\footnote{$t$ -- brezdimenzijski čas, $\beta$ -- recipročna brezdimenzijska temperatura}. Odtod vidimo, da
je $z$ čisto imaginarno število za čas in čisto realno število za temperaturni premik. Oz. drugače povedano

\[
	|\psi(t)\rangle = \e^{-itH}, \qquad |\psi(-i\beta)\rangle = \e^{-\beta H}.
\]

\subsection{Hamiltonian}

Propagator smo dobili tako, da smo Hamiltonian razcepili na operatorja $A$ in $B$, eden delce "`prime"' na sodih,
drugi pa na lihih mestih. Če to matriko zapišemo zgolj za dvodelčno stanje, lahko hamiltonian zapišemo s pomočjo
matrike $A^{(2)}$, ki sedaj deluje le na enem kubitnem paru:

\[
	A^{(2)} = \begin{bmatrix}
		1 &  &  &  \\
		&-1 &\ 2 &  \\
		 & \quad 2 & -1 &  \\
		 &  &  & 1
		\end{bmatrix} = h_{01},
\]

kjer moramo s to matriko izmenično delovati na pare fermionov -- enkrat na vse sode, drugič na vse lihe. Hamiltonian
ima še dele, s katerim delujemo na lihe. Radi bi ga zapisali v isti bazi, kot $h_{01}$, da ga bomo lahko sešteli
in tako imeli le en Hamiltonski blok za dva kubita. Po izračunu dobimo

\[
	h_{10} =  \begin{bmatrix}
		1 &  &  &  \\
		&-1 &\ 2 &  \\
		 & \quad 2 & -1 &  \\
		 &  &  & 1
		\end{bmatrix} = h_{01}
\]

kar lahko z nekaj algebre potrdimo. Po definiciji mora veljati

\begin{equation}
	H^{(2)} = h_{01} + h_{10} = 2h_{01},
\end{equation}

ki ima lastne vrednosti $(-6, 2, 2, 2)$.
Če je naša numerična simulacija pravilna, moramo dobiti torej dobiti $\langle H^{(2)} \rangle_\beta
\stackrel{\beta \to \infty}{\longrightarrow} -6$. Daljše kubitne verige bodo osnovno skalirale proti 0, ker imamo
anti-feromagnet. Recimo, za verigo dolžine $3$ se lahko v potu svojega obraza prebijemo do hamiltoniana

\begin{equation}
	H^{(3)} = \begin{bmatrix}
	3 & 0 & 0 & 0 & 0 & 0 & 0 & 0 \\
	0 & -1 & 2 & 0 & 2 & 0 & 0 & 0 \\
	0 & 2 & -1 & 0 & 2 & 0 & 0 & 0 \\
	0 & 0 & 0 & -1 & 0 & 2 & 2 & 0 \\
	0 & 2 & 2 & 0 & -1 & 0 & 0 & 0 \\
	0 & 0 & 0 & 2 & 0 & -1 & 2 & 0 \\
	0 & 0 & 0 & 2 & 0 & 2 & -1 & 0 \\
	0 & 0 & 0 & 0 & 0 & 0 & 0 & 3
	\end{bmatrix},
\end{equation}

ki se v diagonalni obliki zapiše kot $\text{diag}(-3, -3, -3, -3, 3, 3, 3, 3)$.

Vendar pa je narediti tako matriko težko in se potrati veliko procesorskega časa s Kroneckerjevim produktom,
nato pa še z matričnim množenjem. Vendar pa mi lahko isto naredimo na boljši način. Izkoristimo namreč lahko
to, da je naš vektor $\underline{x}$ v resnici nekakšen spinor

\[
	\underline{x} = x_0|000\rangle + x_1|001\rangle + x_2|010\rangle + x_3|011\rangle + \ldots
\]

komponente pa ima prikladno urejene. Naš hamiltonian je simetričen preko diagonale $H_{00} - H_{NN}$,
vendar pa je simetričen tudi preko diagonale $H_{0N} - H_{N0}$, kar pomeni, da lahko število ničel in
enic v stanju zamenjamo in bo matrični element isti.

Če delujemo z operatorjem na enotski vektor v tem prostoru, ki ga bom zaenkrat označil s $\psi\rangle$,
dobimo

\[
	H|\psi\rangle = \sum_{j=0}^{N-1} \big(2\sigma_j^+\sigma_{j+1}^- + \sigma^z_j\sigma^z_{j+1} + \text{h.c.}\big)
		|\overbrace{00\ldots0}^{n}\overbrace{11\ldots1}^{m}\rangle,
\]

kaj potem pravzaprav dobimo?

Poiščimo najprej diagonalni del, saj je ta lažji. Zaradi simetrije na zamenjavo $0 \leftrightarrow 1$ mora izraz imeti
isto lastnost in ni važno, če so ničle in enice v stanju premešane. Izkaže se, da za $m,n \geq 2$ velja

\[
	H^z|\psi\rangle = \sum_{j=0}^{N-1}\sigma^z_j\sigma^z_{j+1}|\overbrace{0\ldots0}^n \overbrace{1\ldots1}^m\rangle = \bigg[\binom{n}{2} + \binom{m}{2} - mn\bigg]|\psi\rangle,
\]

kjer je

\[
	\binom{n}{k} = \frac{n!}{k!(n-k)!},
\]

nam vsem dobro znani binomski simbol. Izvendiagonalne elemente je poiskati dosti težje, saj bomo dobili mešanje
z večimi komponentami vektorjev.

Če imamo neko enotsko stanje $\psi\rangle$, ga lahko v naši bazi zapišemo kot

\[
	|\psi\rangle = |\overset{0}{\underset{n-1}{s_0}}\ \overset{1}{\underset{n-2}{s_1}}\ldots
		\overset{n-1-j}{\underset{j}{s_{n-1-j}}}\ldots\overset{n-1}{\underset{0}{s_{n-1}}}\rangle,
\]

zgornja številka pomeni kje smo, če štejemo stanje kubita, druga pa pač na katerem, "`bitu"' smo v vektorju stanj.

Ko na ta enotski vektor delujemo z izvendiagonalnim delom hamiltoniana dobimo (h.c. člen bomo upoštevali pozneje)

\[
	H^\pm|\psi\rangle = 2\sum_{j=0}^{N-1}(\sigma^+_j\sigma^-_{j+1})
		|0\ldots0\stackrel{j}{1}0\ldots0\rangle = 2|0\ldots0\stackrel{j+1}{1}0\ldots0\rangle,
\]

če je ta $|\psi\rangle$ ponazarjal enotski vektor $k$-te komponente, potem ga ta operator poveže s
`$k + 2^{n-1-j}$'-to komponento.

To je bil trivialen primer, ko se je to zgodilo enkrat, sicer pa moramo to storiti tolikokrat, kolikokrat se
pojavi par $01$. Števili stanji $0$ in $1$ morata biti sosednji upoštevaje robni pogoj.
Zaradi hermitsko konjugiranega člena bo ista povezava veljala
tudi nazaj, tj.
$k + 2^{n-1-j}$-ti člen bo povezan s $k$-tim, zaradi simetrije skozi drugo diagonalo, se bodo iste povezave
veljale tudi za $k+N/2$-ti člen ter $k+N/2+2^{n-1-j}$-ti člen. Tako ne potrebujemo cele matrike, ampak samo
povezave, kar bi moralo bistveno olajšati množenje s Hamiltonianom, ki je v bistvu razpršena matrika ranga
$2^n$, z nekaj neničelnimi členi. Lahko sicer napovemo koliko jih je. Če je matrika ranga $N$, je št.
neničelnih členov enako $M$,

\[
	M = 2 + \frac{}{}
\]

Za spinsko korelacijo $C(t)$ in korelacijo spinskega toka $C(t)_J$ imamo to smolo, da naš operator ne
komutira s hamiltonianom, to pa pomeni, da ne komutira s propagatorjem, zaradi česar moramo poleg propagiranja
$|\psi(0)\rangle$ propagirati skozi čas tudi $|\chi (0)\rangle = \sigma^z (0)|\psi (0)\rangle$. Še vedno
lahko optimiziramo $\langle \psi(t)|\sigma^z(0)|\chi(t)\rangle$, ki je potem

\[
	\underline{x}^\dagger \sigma^z(0) \underline{y} = \sum_{j = 0}^{N/2 - 1} (x^*_{2j}y_{2j} -
		x^*_{2j+1}y_{2j+1}),
\]

saj je spinski operator sila preprost. Isto bi radi storili še za $J_{j,j+1}$. Zapišimo ga v prikladnejšo obliko:

\[
	J_{j,j+1} = \sigma^x_j\sigma^y_{j+1} - \sigma^y_j\sigma^x_{j+1} =
		2i(\sigma^+_j \sigma^-_{j+1} - \sigma^-_j\sigma^+_{j+1}).
\]

V bazi od prej, lahko analogno zapišemo matriko $J^{(2)}$ kot

\[
	J^{(2)} = -2\begin{bmatrix}
		0 & 0 & 0 & 0 \\
		0 & 0 & -i & 0 \\
		0 & i & 0 & 0 \\
		0 & 0 & 0 & 0
		\end{bmatrix}.
\]

\end{document}

