\documentclass[a4 paper, 12pt]{article}
\usepackage[slovene]{babel}
\usepackage[utf8]{inputenc}
\usepackage[T1]{fontenc}
\usepackage[small, width=0.8\textwidth]{caption}
\usepackage[pdftex]{graphicx}
\usepackage{amssymb, amsmath, fullpage, float, hyperref, stackrel}

\begin{document}

\begin{center}
\textsc{Višje računske metode}\\
\textsc{2012/13}\\[0.5cm]
\textbf{4. naloga -- Kubitne verige}
\end{center}
\begin{flushright}
\textbf{Jože Zobec}
\end{flushright}

\newcommand{\e}{
	\ensuremath{\text{e}}
}

\section{Uvod}

Oznake in definicije bodo iste, kot so v skripti.

V tej nalogi imamo propagacijo po času, kot tudi po temperaturi (imaginarnem času).

Ker bomo imeli opravka z veliko množenji, matrika pa je le dimenzije $4 \times 4$ in je simetrična tridiagonalna
itd. se splača to transformacijo zakodirati naravnost v kodo. Če moramo transformirati vektor
$\underline{x} = (x_0, x_1, x_2, x_3)^T$, se ta po propagaciji s kompleksnim koeficientom $\alpha$ za kompleksni korak
$z$ transformira tako:

\begin{equation}
	\begin{bmatrix}
		x_0 \\
		x_1 \\
		x_2 \\
		x_3
	\end{bmatrix} \stackrel{U^{(2)}(\alpha z)}{\longrightarrow} \text{e}^{-\alpha z}
	\begin{bmatrix}
		\exp(2\alpha z) x_0 \\
		\cosh(2\alpha z)x_1 + \sinh(2\alpha z)x_2 \\
		\sinh(2\alpha z)x_1 + \cosh(2\alpha z)x_2 \\
		\exp(2\alpha z) x_3
	\end{bmatrix}.
	\label{eqn:U}
\end{equation}

Temperaturni propagator je $\sim \exp(-\beta H)$, in časovni je $\sim \exp (-itH)$, kjer sta $\beta$ in $t$ realni
količini\footnote{$t$ -- brezdimenzijski čas, $\beta$ -- recipročna brezdimenzijska temperatura}. Odtod vidimo, da
je $z$ čisto imaginarno število za čas in čisto realno število za temperaturni premik. Oz. drugače povedano

\[
	|\psi(t)\rangle = \e^{-itH}, \qquad |\psi(-i\beta)\rangle = \e^{-\beta H}.
\]

\subsection{Hamiltonian}

Propagator smo dobili tako, da smo Hamiltonian razcepili na operatorja $A$ in $B$, eden delce "`prime"' na sodih,
drugi pa na lihih mestih. Če to matriko zapišemo zgolj za dvodelčno stanje, lahko hamiltonian zapišemo s pomočjo
matrike $A^{(2)}$, ki sedaj deluje le na enem paru fermionov:

\[
	A^{(2)} = \begin{bmatrix}
		1 &  &  &  \\
		&-1 &\ 1 &  \\
		 & \quad 1 & -1 &  \\
		 &  &  & 1
		\end{bmatrix},
\]

kjer moramo s to matriko izmenično delovati na pare fermionov -- enkrat na vse sode, drugič na vse lihe. Vendar pa vemo,
da produkt vseh sodih z vsemi lihimi komutira, kot vidimo že na matriki sami. Zaradi konstrukcije matrike zadošča
za delujemo s centralnim blokom in haramo po matriki, vendar je delovanje na dvodelnčno stanje enako učinkovito. Po
vzoru enačbe~\eqref{eqn:U} napišemo

\begin{equation}
	\begin{bmatrix}
		x_0 \\
		x_1 \\
		x_2 \\
		x_3
	\end{bmatrix} \stackrel{A^{(2)}}{\longrightarrow}
	\begin{bmatrix}
		x_0 \\
		x_2 - x_1 \\
		x_1 - x_2 \\
		x_3
	\end{bmatrix}.
\end{equation}

Tako lahko izračunamo $H|\psi\rangle$ s tem, da z operatorjem $A^{(2)}$ najprej prevozimo vse lihe, nato pa še vse sode
pare. Na ta način smo si močno prikrajšali delo za izračun $\langle\psi | H | \psi\rangle$.

\end{document}

