\documentclass[a4 paper, 12pt]{article}
\usepackage[slovene]{babel}
\usepackage[utf8]{inputenc}
\usepackage[T1]{fontenc}
\usepackage[small, width=0.8\textwidth]{caption}
\usepackage[pdftex]{graphicx}
\usepackage{amssymb, amsmath, fullpage, float, hyperref, stackrel}

\begin{document}

\begin{center}
\textsc{Višje računske metode}\\
\textsc{2012/13}\\[0.5cm]
\textbf{4. naloga -- Kubitne verige}
\end{center}
\begin{flushright}
\textbf{Jože Zobec}
\end{flushright}

\section{Uvod}

Oznake in definicije bodo iste, kot so v skripti.

V tej nalogi imamo propagacijo po času, kot tudi po temperaturi (imaginarnem času).

Ker bomo imeli opravka z veliko množenji, matrika pa je le dimenzije $4 \times 4$ in je simetrična tridiagonalna
itd. se splača to transformacijo zakodirati naravnost v kodo. Če moramo transformirati vektor
$\underline{x} = (x_0, x_1, x_2, x_3)^T$, se ta po propagaciji s kompleksnim koeficientom $\alpha$ za kompleksni čas
$z$ transformira tako:

\begin{equation}
	\begin{bmatrix}
		x_0 \\
		x_1 \\
		x_2 \\
		x_3
	\end{bmatrix}' \stackrel{U^{(2)}(\alpha z)}{\longrightarrow} \text{e}^{-\alpha z}
	\begin{bmatrix}
		\exp(2\alpha z) x_0 \\
		\cosh(2\alpha z)x_1 + \sinh(2\alpha z)x_2 \\
		\sinh(2\alpha z)x_1 + \cosh(2\alpha z)x_2 \\
		\exp(2\alpha z) x_3
	\end{bmatrix}.
\end{equation}

\end{document}

