\documentclass[a4 paper, 12pt]{article}
\usepackage[slovene]{babel}
\usepackage[utf8]{inputenc}
\usepackage[T1]{fontenc}
\usepackage[small, width=0.8\textwidth]{caption}
\usepackage[pdftex]{graphicx}
\usepackage{amssymb, amsmath, fullpage, float, hyperref}

\begin{document}

\begin{center}
\textsc{Višje računske metode}\\
\textsc{2012/13}\\[0.5cm]
\textbf{2. naloga -- stacionarni problem Schr\" odingerjeve enačbe}
\end{center}
\begin{flushright}
\textbf{Jože Zobec}
\end{flushright}

\section{Uvod}

V tej nalogi iščemo lastne funkcije prejšnje naloge. Problem bi radi reševali v lastni bazi,
ker se problem v tem primeru razcepi na časovni del, ki ga znamo rešiti (rešitev je
$\exp(-iE_n)$) in na krajevni del, ki je rešitev amplitudne enačbe.

Lastne baze $\big\{|\psi_n\rangle\big\}$ ne poznamo, vendar lahko problem vseeno še vedno
rešujemo v neperturburani bazi $\big\{|\phi_n\rangle\big\}$. Vsako funkcijo $|\psi_n\rangle$
lahko potem zapišemo kot

\[
	|\psi_n\rangle = \sum_{m=0}^\infty v_{nm}|\phi_m\rangle.
\]

V nadaljevanju bom upošteval Einsteinovo konvencijo. Prav tako ne bom posvečal posebne
pozornosti razliki med npr. matriko $A$ ter njenimi komponentami $A_{ij}$, upam da bo razvidno
iz konteksta. Matrika $v_{mn}$ mora biti unitarna, saj povezuje dve ortonormalni bazi v istem prostoru.
To lahko pokažemo tako:

\begin{align}
	\langle\psi_n|\psi_m\rangle &= \delta_{mn}, \notag \\
	\langle\phi_a| v^*_{an} v_{mb} |\phi_b\rangle &= \delta_{nm}. \label{norm}
\end{align}

Za vse primere, razem $m = n$ je naša vsota enaka nič, zato je zadosti, da obravnavamo le 
$m = n$.

\begin{equation}
	\langle\phi_a| (v^\dagger v)_{ab} |\phi_b\rangle = 1. \\
	\label{stuff}
\end{equation}

Matriko $(v^\dagger v)_{ab}$ razvijemo po komponentah na diagonalni in izvendiagonalni del:

\begin{equation*}
	(v^\dagger v)_{ab} = \alpha\delta_{ab} +
		\beta\underbrace{\langle\phi_i| v^*_{ia}v_{bj} |\phi_j\rangle}_{=\delta_{ab}},
\end{equation*}

po predpostavki~\eqref{norm} vidimo, da v bistvu nimamo izvendiagonalnega člena, zadošča že
diagonalni parameter $\alpha$, kar pomeni $(v^\dagger v)_{ab} = \alpha \delta_{ab}$. To
vstavimo v en.~\eqref{stuff} in dobimo

\begin{equation*}
	\langle\phi_a|\alpha\delta_{ab}|\phi_b\rangle = \alpha\langle\phi_a|\phi_b\rangle = 1,
\end{equation*}

vendar pa je baza ${|\phi_n\rangle}$ spet ortonormirana, od koder sledi $\alpha = 1$ in

\begin{equation}
	(v^\dagger v)_{ab} = \delta_{ab},
\end{equation}

torej je matrika $v$ unitarna. Odtod tudi sledi, da si matriko $v_{nm}$ mislimo kot lastni
vektor $\underline{v}_n$ matrike $H_{ij}$, saj je $\{\underline{v}_n\}$ spet
ortonormalna baza vektorskega prostora, vendar je to tokrat baza prostora, v katerem "`živi"'
$H_{ij}$. 

\[
	\underline{v}_i \cdot \underline{v}_j = v_{ik} v_{kj}^* = (vv^\dagger)_{ij} = \delta_{ij}.
\]

Matriko $v_{mn}$ lahko dobimo tako: definirajmo projekcijski operator
$\mathbf{P} = |\phi_n\rangle\langle\phi_n|$. To je očitno projekcijski operator, saj
$\mathbf{P}^2 = \mathbf{P}$:

\[
	\mathbf{P}^2 = |\phi_i\rangle\langle\phi_i|\phi_j\rangle\langle\phi_j| =
		|\phi_i\rangle\langle\phi_j|\delta_{ij} = |\phi_i\rangle\langle\phi_i| = 
		\mathbf{P}
\]

Bazo $\big\{|\psi_n\rangle\big\}$ lahko zdaj proiciramo na $\big\{|\phi_n\rangle\big\}$.
Ker proiciramo na isti prostor ne bomo izgubili informacije.

\[
	\mathbf{P}|\psi_m\rangle = |\phi_n\rangle\underbrace{\langle\phi_n|\psi_m\rangle}_{=
		v_{mn}} = v_{mn}|\phi_n\rangle.
\]

Operator $\mathbf{P}$ je po definiciji Fouriejeva transformacija\footnote{Vendar ne
trigonometrična.}, ki nam pomaga rotirati baze v Hilbertovem prostoru. Ta operator je v bazi
$|\phi_n\rangle$ kar identiteta, saj vektorjem $|\phi_n\rangle$ nič ne naredi. Sedaj lahko
lažje razumemo, zakaj je $v$ unitarna matrika:

\begin{align*}
	(v^\dagger v)_{ab} &= \langle\phi_a|\psi_j\rangle^\dagger \langle\phi_b|\psi_j\rangle, \\
	&= \langle\psi_j|\phi_a\rangle \langle\phi_b|\psi_j\rangle, \\
	&= \langle\phi_b\underbrace{|\psi_j\rangle\langle\psi_j|}_{= 1}\phi_a\rangle \\
	&= \langle\phi_a|\phi_b\rangle = \delta_{ab}.
\end{align*}

Pokažimo, da je $\underline{v}_n$ res spet lastni vektor matrike $H_{ij}$. Tokrat bom delal razloček: $H_{ij}$
je komponenta matrike $\mathbf{H}$. Kot sem povedal prej, bomo predpostavili, da je so $|\psi_n\rangle$ lastna baza
za $\mathbf{H}$. Po komponentah se ga v harmonski bazi zapiše kot

\[
	H_{ij} = \langle\phi_i|\hat{H}|\phi_j\rangle, \\
\]

v tej bazi ni diagonalna. Za lastni vektor $\underline{v}_n$ matrike $\mathbf{H}$ velja (brez Einsteinovega načela)

\[
	\mathbf{H}\underline{v}_n = \sum_{i,j} H_{ij}v_{nj} = E_n \sum_{i} v_{ni} = E_n \underline{v}_n.
\]

Preverimo:

\begin{align}
	\sum_{i,j} H_{ij}v_{nj} &= \sum_i \langle\phi_i|\hat{H}\overbrace{\sum_j |\phi_j\rangle\langle\phi_j|}^{= 1}\psi_n\rangle \notag \\
	&= \sum_i \langle\phi_i|\hat{H}|\psi_n\rangle = \sum_i \langle\phi_i|E_n|\psi_n\rangle = E_n \sum_i \langle\phi_i|\psi_n\rangle \notag \\
	&= E_n \sum_i v_{ni} = E_n \underline{v}_n. \label{dokaz}
\end{align}

Vidimo, da je je vektor $\underline{v}_n$ res lastni vetkor matrike $\mathbf{H}$, oz. vrstice matrike $v_{ij}$ so
lastni vektorji.

Kako poiskati matriko $v_{ij}$? Matriko $H_{ij}$ znamo zapisati v bazi $\big\{|x_n\rangle\big\}$ iz prejšnje naloge.
Matrika $v_{ij}$ je pač matrika, ki ta hamiltonian diagonalizira:

\[
	\langle\psi_i|\hat{H}|\psi_j\rangle = \langle\phi_a|\hat{v}^\dagger\hat{H}\hat{v}|\phi_b\rangle,
\]

oz.

\[
	H^{(\psi)}_{ij} = v_{ia}^* H^{(\phi)}_{ab} v_{bj} = E_i\delta_{ij}.
\]

Za hamiltonian lahko vzamemo matriko iz diferenčne sheme iz prejšnje naloge. V prejšnji nalogi smo videli, da
tridiagonalna diferenčna shema ni dovolj dobra, zaradi česar se bomo poslužili višjih shem, vendar pa zahtevnost
za diagonalizacijo takih matrik raste! K sreči imamo na voljo Lanczosev algoritem, s katerim lahko vsako Hermitsko matriko
pretvorimo v tridiagonalno Hermitsko matriko. Algoritmi za polne matrike dimenzije $N$ zahtevajo $\mathcal{O}(N^3)$
operacij, za tridiagonalne matrike pa le $\mathcal{O}(N^2)$ -- Lanczosov algoritem se izplača, ima pa isto zahtevnost.

Naša diagonalizacija poteka torej prek dveh korakov. Naš hamiltonian je realna simetrična matrika, torej pojem
unitarnosti postane ortogonalnost. Kvadratno formo naše matrike zapišemo kot

\[
	x^TAx = x^T \underbrace{L^T L}_{= 1} A \underbrace{L^T L}_{= 1} x,
\]

kjer je $L$ matrika Lanczosove transformacije -- rotira spet v ortonormalno bazo istega prostora, torej mora biti ortogonalna, kot zgornji
izraz tudi predpostavki. Ker je ta matrika ortogonalna pomeni da ohranja spekter matrike $A$. V tej bazi je matrika $A$ tridiagonalna.
Izvesti moramo še diagonalizacijo z matriko $V$

\begin{align*}
	x^T L^T L A L^T L x &= x^T L^T V^T V L A L^T V^T V L x \\
		&= \underbrace{(VLx)^T}_{y^T} \overbrace{\underbrace{(VL)}_{v} A \underbrace{(VL)^T}_{v^T}}^{D} \underbrace{(VLx)}_{y} \\
		&= y^T D y,
\end{align*}

kjer je matrika $D$ diagonalna, pretvorniška matrika pa je $v_{ij} = V_{ik}L_{kj}$. Matriki $V^T$ in $L^T$ vsebujeta bazne
vektorje po stolpcih, oz. matrika $v$ vsebuje lastne vektorje po vrsticah, kar sem demonstriral malo prej v
enačbi~\eqref{dokaz}.

Namesto, da bi diagonalizirali hamiltonian, lahko numerično rešujemo amplitudno enačbo ter dako dobimo lastne pare prek
sekantne metode.

Matrika iz prejšnje naloge NI v bazi $\big\{|\phi_n\rangle\big\}$, ampak v $\big\{|x_n\rangle\big\}$, tako da

\[
	\langle \phi_n | x_m \rangle = \phi_{n,m} = \phi_n (x_m) = \phi_n (m \cdot h), \quad m \in \mathbb{Z}, \quad h \in \mathbb{R}.
\]

Spet lahko definiramo

\[
	\underline{\phi}_n = \sum_m \phi_n(x_m) = \sum_m \phi_{n,m}.
\]

Ta baza je zelo prikladna, saj lahko opazujemo propagacijo kakršnega valovnega paketa. Lahko spet naredimo tudi koherentno stanje
in preverimo delovanje hamiltoniana. Prav tako lahko matriko $\phi_{nm}$ uporabimo za rotacijo iz prostora baze 
$\big\{|x_n\rangle\big\}$ v bazo $\big\{|\phi_n\rangle\big\}$. To matriko poznamo, to so kar lastne funkcije harmonskega
oscilatorja.

Vendar pa ta baza ni dobra iz dveh razlogov:
\begin{enumerate}
	\item{Hermitovi polinomi so ortogonalni v zveznem prostoru -- ortogonalnst se z diskretizacijo močno pokvari za polinome
		visokih redov. Potem sploh ni važno, če računamo z diferenčno shemo $\mathcal{O}(h^8)$, če je ena izmed rotacijskih
		matrik slabo ortogonalna. Torej moramo za pridobitev te matrike še enkrat prej izvesti diagonalizacijo za dobro
		ortogonalnost, kar pa je potrata časa.}
	\item{Če pričnemo v taki bazi ne vemo kako zapisati koherentno stanje. Seveda, lahko ga zapišemo v bazi $\big\{|x_n\rangle\big\}$,
		in ga potem transformiramo v $\big\{|\phi_n\rangle\big\}$, vendar je to potrata časa.}
\end{enumerate}

Če iz baze $x_n$ rotiramo v $\phi_n$, nato odtod v Lanczosovo bazo in na koncu v lastno bazo $\psi_n$ se naša matrika $\psi_{ij}$, ki
povezuje $\big\{|x_n\rangle\big\}$ in $\big\{|\psi_n\rangle\big\}$ zapiše kot

\[
	\psi_{ij} = v_{ik} \phi_{kj} = V_{il} L_{lk} \phi_{kj},
\]

tako dobimo $\psi_{n,m} = \psi_n (x_m)$ in pa lastne vektorje $\underline{\psi}_n$.

Simetrične tridiagonalne matrike lahko diagonaliziramo z metodo MRRR\footnote{Multiple Relatively Robust Representations}, ki za
diagonalizacijo $k$ lastnih vrednosti matrike ranga $N$ potrebuje le $\mathcal{O} (kN)$ operacij, kar pomeni da ima ta hip
najmanj slabo časovno zahtevnost. V primeru, ko želimo izračunati vse lastne vrednosti ima zahtevnost $\mathcal{O}(N^2)$ za
matrike ranga $N$. Uporabil sem rutino {\tt DSTEMR} iz knjižnjice {\sc Lapack}, ki skuša avtomatsko doseči preciznost na
osem decimalnih mest.

Število točnih lastnih vrednosti ni odvisno le od $N$, pač pa tudi od $h$, oz. bolje, $h$ v resnici ni neodvisen parameter,
pač pa je odvisen od $N$.

S tem, ko smo naš problem diskretizirali in zaprli v neko škatlo, smo neskončno kvantnih stanj stlačili v nek končen prostor.
Če je $h$ prevelik bomo dobili podvajanje lastnih vrednosti (energij), ker naš diskretni prostor ne bo imel dovolj prostostnih
stopenj. Izkaže se, da $h$ prav tako ne sme biti premajhen. Potem se naša fina skala ne pozna, zadušijo jo visoke energije, ki
pa so izračunane z veliko preciznostjo.

Za vsak $N$ torej obstaja nek optimalni $h$, ki najbolje opiše naš problem. Dobimo ga prek minimizacije -- za nek $N$ mora biti
razmerje slabih lastnih vrednosti proti dobrim lastnim vrednostim najmanjše. Sicer lahko to s poskušanjem bolj ali manj uganemo,
vendar to traja kar veliko časa. To lahko za nas stori tudi računalnik.

Parameter $h$ ima skrito odvisnost od $N$ zaradi tega, ker z njim omogočamo optimalno ortogonalnost Lanczosove matrike.

\end{document}
