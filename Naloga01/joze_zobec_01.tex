\documentclass[a4 paper, 12pt]{article}
\usepackage[slovene]{babel}
\usepackage[utf8]{inputenc}
\usepackage[T1]{fontenc}
\usepackage[small, width=0.8\textwidth]{caption}
\usepackage[pdftex]{graphicx}
\usepackage{amssymb, amsmath, fullpage, float, hyperref}

\begin{document}

\begin{center}
\textsc{Višje računske metode}\\
\textsc{2012/13}\\[0.5cm]
\textbf{1. naloga -- časovni razvoj Schr\" odingerjeve enačbe}
\end{center}
\begin{flushright}
\textbf{Jože Zobec}
\end{flushright}

\section{Uvod}

Pri tej nalogi smo delali z diskretnomi metodami. To pomeni, da smo za naš problem prostor
predstavili na mreži in na tej mreži aproksimirali odvode. Aproksimacije odvodov s končnimi
diferencami so sicer lahko nestabilne, saj za majhne korake divergirajo. Vendar, če vzamemo
dovolj človeške vrednosti, nas to ne bo preveč motilo.

Na voljo imamo več metod. Sam sem preizkusil le eno, to je implicitno metodo za reševanje.
Implicitne metode so v veliki večini primerov stabilnejše, zato nisem tratil časa s tem, da bi
to resnico znova\footnote{tj. kot pri Matematično-fizikalnem praktikuumu in Modelski analizi, Numerične metode \ldots}
potrdil.

Čas sem raje vložil v optimizacijo svojega programa.

Oznake so iste, kot v profesorjevi skripti, razen če eksplicitno piše kako drugače. Tako je
na primer $V_m = V(mh)$ in $V = V(x)$ je potencial, $h$ je krajevni korak itd.

\section{Naloga}

Imamo potencial

\begin{equation}
	V(x) = \frac{1}{2}x^2 + \lambda x^4,
\end{equation}

za začetno stanje pa vzamemo eno od lastnih stanj harmonskega oscilatorja. Problem je
zanimivejši, seveda, če ta valovni paket izmaknemo iz ravnovesne lege, tj.

\[
	\Psi_N (x, t = 0) = \phi_N (x - \alpha),
\]

kjer je $\phi_N$ prej omenjeno lastno stanje harm. oscilatorja.

Parameter $\lambda$
lahko poljubno spreminjamo, saj ostane neodvisen na našo brezdimenzijsko transformacijo.

Imamo Dirichletove robne pogoje, kar pomeni, da z implicitno metodo v bistvu rešujemo sistem
\[
	A_{ij}x_j = b_i,
\]

kjer je $A$ tridiagonalna matrika. Tak sistem lahko rešujemo v $\mathcal{O}(N)$.

\subsection{Vprašanja}
\begin{itemize}
	\item{Naš problem smo zaprli v škatlo dimenzije $L$. Kako velik mora biti ta $L$,
		da bo rešitev zaradi tega zanemarljivo pokvarjena? ($\alpha = 0$)}
	\item{Kako se spreminja rešitev $\Psi (x,t; \lambda)$ s parametrom $\lambda$?
		($\alpha = 0$)}
	\item{Poglejmo, kaj se zgodi z valovnim paketom po vklopu anharmonske motnje $\lambda$.
		($\alpha \neq 0$)}
\end{itemize}

\subsection{Metoda}

Implicitno metodo izvajamo s pomočjo iteracije. Opazimo, da lahko našo rešitev obravnavamo
kot vektor, metodo samo pa kot linearni operator, oz. matriko.

Če prepišemo sistem iz profesorjeve skripte\footnote{gornji indeksi $n$ predstavljajo časovni
indeks, prav tako v skripti nekje manjka faktor $1/2$, kar sem tu popravil}

\begin{align*}
	&\psi_m^{n+1} - \frac{i\tau}{2}\bigg\{\frac{1}{2h^2}\left(\psi_{m+1}^{n+1} +
		\psi_{m-1}^{n+1} - 2\psi_m^{n+1}\right) - V_m\psi_m^{n+1}\bigg\} = \\
		&= \psi_m^n + \frac{i\tau}{2}\bigg\{\frac{1}{2h^2}\left(\psi_{m+1}^n +
		\psi_{m-1}^n - 2\psi_m^n\right) - V_m\psi_m^n\bigg\}
\end{align*}

ugotovimo, da je ekvivalenten sistemu

\begin{equation}
	L_+ \mathbf{\Psi}^{(n+1)} = L_- \mathbf{\Psi}^{(n)} = \mathbf{b}^{(n)},
\end{equation}

kjer je matrika $L_+$ enaka

\begin{equation}
	L_\pm = 1 \mp \frac{i\tau}{4h^2}\begin{bmatrix}
		a_0 & 1 & 0 & \ldots & & & 0\\
		1 & a_1 & 1 & 0 & \ldots & & 0\\
		0 & 1 & a_2 & 1 & 0 & \ldots & 0\\
		\vdots & & & & \ddots \\
		& & & \ldots & 0 & 1 & a_{M-1} \end{bmatrix}, \qquad
	a_k = -2 - 2h^2V_k,
\end{equation}

vektorji $\mathbf{\Psi}^{(n)}$ pa so stolpci

\begin{equation}
	\mathbf{\Psi}^{(n)} = \begin{bmatrix} \psi_0^n \\ \psi_1^n \\ \vdots \\ \psi_{M-1}^n
		\end{bmatrix}.
\end{equation}

Količin z indeksom $n+1$ ne poznamo, od koder se vidi iteracijski značaj naše metode. Problem
Schr\" odingerjeve enačbe smo tako prevedli na reševanje tridiagonalnega sistema.

Videti je, da metodo dobimo poceni, vendar moramo časovno stabilnost drago plačati, zaradi
pogoja $h^2 \gg \tau$. To lahko izboljšamo s tem, da drugi odvod aproksimiramo z višjim redom
diferenc, npr. petim redom. V tem primeru imamo pasato matriko s petimi "`diagonalami"'.

Naša enačba se potem glasi

\begin{align*}
	&\psi_m^{n+1} - \frac{i\tau}{2}\bigg\{\frac{1}{2h^2}\left[\frac{4}{3}\left(
		\psi_{m+1}^{n+1} + \psi_{m-1}^{n+1}\right) - \frac{1}{12}\left(
		\psi_{m+2}^{n+1} + \psi_{m-2}^{n+1}\right) - \frac{5}{2}\psi_m^{n+1}\right] 
		- V_m\psi_m^{n+1}\bigg\} = \\
	&= \psi_m^n + \frac{i\tau}{2}\bigg\{\frac{1}{2h^2}\left[\frac{4}{3}\left(
		\psi_{m+1}^n + \psi_{m-1}^n\right) - \frac{1}{12}\left(
		\psi_{m+2}^n + \psi_{m-2}^n\right) - \frac{5}{2}\psi_m^n\right] -
		V_m\psi_m^n\bigg\},
\end{align*}

v tem primeru si lahko privoščimo nižje med razmerje $\tau$ in $h^2$. V tem primeru
dobimo

\begin{equation}
	L_\pm = 1 \mp \frac{i\tau}{4h^2}\begin{bmatrix}
		a_0 & 4/3 & -1/12 & 0 & \ldots & \ldots & 0 \\
		4/3 & a_1 & 4/3 & -1/12 & 0 & \ldots & 0 \\
		-1/12 & 4/3 & a_2 & 4/3 & -1/12 & \ldots & 0 \\
		\vdots & & & & \ddots & \\
		0 & \ldots & \ldots & 0 & -1/12 & 4/3 & a_{M-1} \end{bmatrix}, \qquad
	a_k = -\frac{5}{2} - 2h^2V_m,
	\label{metoda}
\end{equation}

kar pride zelo prav, saj je na primer dobro razmerje že $\tau = 0.002$ in $h = 0.1$ za prvih
$\sim 1000$ iteracij.

\subsection{Natančnost}

Natančnost je bilo treba izbrati tako, da je bilo razmerje med hitrostjo in kvaliteto računa
ravno pravšnje. Po daljšem preigravanjau sem uporabil kar prej omenjeni vrednosti, tj.
\fbox{$\tau = 0.002$ in $h = 0.1$}. Poizkušal sem s tridiagonalno matriko, vendar za moje
potrebe ni bila dovolj natančna, oz. računalo je počasi, zaradi tega, ker je bilo treba vzeti
izjemno majhen časovni korak, da se je videla ena perioda pa je bilo treba izračunati več
tisoč iteracij. Za moj primer je bilo $N = 8000$ ravno prava številka, kjer je $N$ maksimalno
število časovnih iteracij. Prostor sem razdelil na $M = 300$ enakih koščkov, kar po mojem
zadošča tekom celotnega poročila.

\subsection{Karakteristična dolžina}

Kakšen $L$ mora biti, da bodo efekti "`škatle"' zanemarljivi? Kvadrat valovnega paketa je
neničelen skoraj vsepovsod (tj. povsod, razen na končnem številu točk). Vendar pa lahko
vzamemo za merilo nek $\epsilon \in \mathbb{R}^+$. Kvadrat norme valovnega paketa,
$|\Psi(x,t)|^2$ pada, ko gremo proti neskončnosti (saj spada med integrabilne funkcije).
Če torej vzamemo limito, s katero poiščemo prvo število $a$, za katerega bo veljalo

\[
	\lim_{x \nearrow a} |\Psi (x, t)|^2 = \epsilon,
\]

in jo pošljemo v računalnik, vidimo, da je
$\chi = |\langle \Psi \rangle - a|$  karakteristična velikost našega valovnega paketa.
Da bo naš program deloval dobro, mora veljati \fbox{$L > \chi$}, efekti roba škatle so
$\mathcal{O}(\epsilon)$. Če je $\epsilon$ dovolj majhen škatle praktično ni. Kadar imamo
še odmik iz ravnovesne lege $\alpha$ mora veljati

\[
	\chi + \alpha < L,
\]

kjer so vplivi "`škatle"' spet reda $\epsilon$. Upam, da sem s tem odgovoril na prvo vprašanje.

\section{Rezultati}

\subsection{Ravnovesna lega}

Na začetku samo testiramo stabilnost naše metode, zato pokažemo nekaj osnovnih načinov, ter
njihovo nespremenjljivost. K grafom prilagam tudi animacije, za boljšo oceno. Te animacije
niso zanimive. Imena datotek nosijo hiperpovezave, za lažji dostop do animacije. Za 
predvajanje priporočam {\tt Mplayer}.

\begin{figure}[H]
	\centering
	\includegraphics{phi0_N8000_M300_0_0}
	\vspace{-20pt}
	\caption{Osnovni način, $\lambda=0$. Tu je bila uporabljena tridiagonalna matrika, ker
		vplivi potenciala niso veliki. Datoteka
		\href{phi0_N8000_M300_0_0.avi}{{\tt phi0\_N8000\_M300\_0\_0.avi}}
		vsebuje video.}
	\label{fig1}
	\vspace{-10pt}
\end{figure}

\begin{figure}[H]
	\centering
	\includegraphics{phi1_N8000_M300_0_0}
	\vspace{-20pt}
	\caption{Prvo vzbujeno stanje, $\lambda=0$. Datoteka
		\href{phi1_N8000_M300_0_0.avi}{{\tt phi1\_N8000\_M300\_0\_0.avi}}
		vsebuje video.}
	\label{fig2}
	\vspace{-10pt}
\end{figure}

\begin{figure}[H]
	\centering
	\includegraphics{phi2_N8000_M300_0_0}
	\vspace{-20pt}
	\caption{Drugo vzbujeno stanje, $\lambda=0$. Datoteka
		\href{phi2_N8000_M300_0_0.avi}{{\tt phi2\_N8000\_M300\_0\_0.avi}}
		vsebuje video.}
	\label{fig3}
	\vspace{-10pt}
\end{figure}

Zanimivo se mi zdi prvo vzbujeno stanje (nima preveč vrhov, niti premalo). Poglejmo, kako
se spreminja, če spreminjam parameter $\lambda$. Od sedaj bom jemal 5-diagonalno matriko,
da bodo rezultati bolj gotovi\footnote{nočem, da "`dihanje"' valovnega paketa pride od
diskretizacijske napake}.

\begin{figure}[H]
	\centering
	\includegraphics{phi1_N8000_M300_1_0}
	\vspace{-20pt}
	\caption{Prvo vzbujeno stanje, $\lambda=0.01$. Za vsak slučaj sem vzel
		raje petdiagonalno matriko. Datoteka
		\href{phi1_N8000_M300_1_0.avi}{{\tt phi1\_N8000\_M300\_1\_0.avi}}
		vsebuje video, na katerem lahko vidimo pulz valovnega paketa.}
	\label{fig4}
	\vspace{-10pt}
\end{figure}

\begin{figure}[H]
	\centering
	\includegraphics{phi1_N8000_M300_5_0}
	\vspace{-20pt}
	\caption{Prvo vzbujeno stanje, $\lambda=0.05$. Datoteka
		\href{phi1_N8000_M300_5_0.avi}{{\tt phi1\_N8000\_M300\_5\_0.avi}}
		vsebuje video, na katerem lahko vidimo pulz valovnega paketa.}
	\label{fig5}
	\vspace{-10pt}
\end{figure}

\begin{figure}[H]
	\centering
	\includegraphics{phi1_N8000_M300_10_0}
	\vspace{-20pt}
	\caption{Prvo vzbujeno stanje, $\lambda=0.1$. Datoteka
		\href{phi1_N8000_M300_10_0.avi}{{\tt phi1\_N8000\_M300\_10\_0.avi}}
		vsebuje video. Pulz valovnega paketa je še izrazitejši.}
	\label{fig6}
	\vspace{-10pt}
\end{figure}

Vidimo, da je naš sistem zelo občutljiv že na majne $\lambda$ -- seveda, če smo predaleč od
izhodišča nam majhen $\lambda$ ne pomaga, saj potencial raste kot $\sim x^4$, torej bi bil
perturbacijski primer veljaven ne le za majhne $\lambda$, ampak bi morali biti tudi dovolj
blizu izhodišča.

\subsection{Izven ravnovesja, tj. $\alpha \neq 0$}

Končno pridemo do zanimivega dela, ko je $\alpha \neq 0$. Za neničelne $\lambda$ se nam valovni
paket razsuva, za $\lambda = 0$ pa mora ostati stabilen, kljub motnji. Pa poglejmo, če to drži.

Za mojo izbiro granulacije $\tau$, $h$, tridiagonalna matrika ni zadostna, zato sem delal z
matrikami iz en.~\eqref{metoda}.

V sledečih grafih vidimo, da se naši valovi obnašajo kot nekakšen puding -- videti so mehki.

\begin{figure}[H]
	\centering
	\includegraphics{phi1_N8000_M300_0_5}
	\vspace{-20pt}
	\caption{Prvo vzbujeno stanje, $\lambda=0$. Datoteka
		\href{phi1_N8000_M300_0_5.avi}{{\tt phi1\_N8000\_M300\_0\_5.avi}}
		vsebuje video. Rahlo se poznajo defekti zaradi diskretizacije.}
	\label{fig7}
	\vspace{-10pt}
\end{figure}

\begin{figure}[H]
	\centering
	\includegraphics{phi1_N8000_M300_1_5}
	\vspace{-20pt}
	\caption{Prvo vzbujeno stanje, $\lambda=0.01$. Datoteka
		\href{phi1_N8000_M300_1_5.avi}{{\tt phi1\_N8000\_M300\_1\_5.avi}}
		vsebuje video. Val se zelo lepo in gladko razsuje. Upam, da se tu
		diskretizacijske napake ne pozna preveč.}
	\label{fig8}
	\vspace{-10pt}
\end{figure}

\begin{figure}[H]
	\centering
	\includegraphics{phi1_N8000_M2500_10_5}
	\vspace{-20pt}
	\caption{Prvo vzbujeno stanje, $\lambda=0.1$. Datoteka
		\href{phi1_N8000_M2500_10_5.avi}{{\tt phi1\_N8000\_M1500\_10\_5.avi}}
		vsebuje video. Za ta primer sem vzel $\tau = 8\cdot10^{-4}$ in
		$h = 0.01$, saj je potencial previsok za dan $h$. Začetni val še prej razpade.
		Časovni korak vseeno ni zadosti majhen in proti koncu se metoda sesuva.}
	\label{fig9}
	\vspace{-10pt}
\end{figure}

Pri grafu~\ref{fig9} sem moral časovno in krajevno skalo bolj fino narezati, ker je bil
potencial prevelik za dan $h$. Če pa hočem zmanjšati $h$ je treba zmanjšati tudi $\tau$.
Valovni paket veliko bolj hitro pade proti središču in se zaleti v nasprotni rob. Vsi prejšnji
grafi so imeli za 16 brezdimenzijskih časovnih enot propagacije, ta pa je imel le za 6 in
kljub temu tako hitro prileti od mesta.

Sicer pa se doagaja točno tisto, kar smo pričakovali -- valovni paket je relativno konstanten 
in stabilen za $\lambda = 0$, saj je lastna valovna funkcija Hamiltonovega operatorja.

Ko vključimo motnjo paket ni več lastno stanje, ampak superpozicija le teh. To stanje se
razstavi po teh lastnih načinih za katere pa velja disperzija -- vsak izmed teh načinov se
propagira z drugačno hitrostjo.

\section{Zaključek}

Implicitna metoda se dobro obnese in je za ustrezno izbiro granulacije stabilna za več tisoč
korakov. Metoda klecne če smo predaleč od izhodišča, ker postane potencial previsok in
napake pri množenju rastejo.

\end{document}

