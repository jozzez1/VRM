\documentclass[a4 paper, 12pt]{article}
\usepackage[slovene]{babel}
\usepackage[utf8]{inputenc}
\usepackage[T1]{fontenc}
\usepackage[small]{caption}
\usepackage[pdftex]{graphicx}
\usepackage{amssymb, amsmath, fullpage}

\begin{document}

\begin{center}
\textsc{Višje računske metode}\\
\textsc{2012/13}\\[0.5cm]
\textbf{1. naloga -- časovni razvoj Schr\" odingerjeve enačbe}
\end{center}
\begin{flushright}
\textbf{Jože Zobec}
\end{flushright}

\section{Uvod}

Pri tej nalogi smo delali z diskretnomi metodami. To pomeni, da smo za naš problem prostor
predstavili na mreži in na tej mreži aproksimirali odvode. Aproksimacije odvodov s končnimi
diferencami so sicer lahko nestabilne, saj za majhne korake divergirajo. Vendar, če vzamemo
dovolj človeške vrednosti, nas to ne bo preveč motilo.

Na voljo imamo več metod. Sam sem preizkusil le eno, to je implicitno metodo za reševanje.
Implicitne metode so v veliki večini primerov stabilnejše, zato nisem tratil časa s tem, da bi
to resnico znova\footnote{tj. kot pri Matematično-fizikalnem praktikuumu in Modelski analizi, Numerične metode \ldots}
potrdil.

Čas sem raje vložil v optimizacijo svojega programa.

Oznake so iste, kot v profesorjevi skripti, razen če eksplicitno piše kako drugače. Tako je
na primer $V_m = V(mh)$ in $V = V(x)$ je potencial, $h$ je krajevni korak itd.

\section{Naloga}

Imamo potencial

\begin{equation}
	V(x) = \frac{1}{2}x^2 + \lambda x^4,
\end{equation}

za začetno stanje pa vzamemo eno od lastnih stanj harmonskega oscilatorja. Problem je
zanimivejši, seveda, če ta valovni paket izmaknemo iz ravnovesne lege, tj.

\[
	\Psi_N (x, t = 0) = \phi_N (x - \alpha),
\]

kjer je $\phi_N$ prej omenjeno lastno stanje harm. oscilatorja.

Parameter $\lambda$
lahko poljubno spreminjamo, saj ostane neodvisen na našo brezdimenzijsko transformacijo.

Imamo Dirichletove robne pogoje, kar pomeni, da z implicitno metodo v bistvu rešujemo sistem
\[
	A_{ij}x_j = b_i,
\]

kjer je naš $A$ tridiagonalna matrika. Tak sistem lahko rešujemo v $\mathcal{O}(N)$.

\subsection{Vprašanja}
\begin{itemize}
	\item{Naš problem smo zaprli v škatlo dimenzije $L$. Kako velik mora biti ta $L$,
		da bo rešitev zaradi tega zanemarljivo pokvarjena? ($\alpha = 0$)}
	\item{Kako se spreminja rešitev $\Psi (x,t; \lambda)$ s parametrom $\lambda$?
		($\alpha = 0$)}
	\item{Poglejmo, kaj se zgodi z valovnim paketom po vklopu anharmonske motnje $\lambda$.
		($\alpha \neq 0$)}
\end{itemize}

\subsection{Metoda}

Implicitno metodo izvajamo s pomočjo iteracije. Opazimo, da lahko našo rešitev obravnavamo
kot vektor, metodo samo pa kot linearni operator, oz. matriko.

Če prepišemo sistem iz profesorjeve skripte\footnote{gornji indeksi $n$ predstavljajo časovni
indeks, prav tako v skripti nekje manjka faktor $1/2$, kar sem tu popravil}

\begin{align*}
	&\psi_m^{n+1} - \frac{i\tau}{2}\bigg\{\frac{1}{2h^2}\left(\psi_{m+1}^{n+1} +
		\psi_{m-1}^{n+1} - 2\psi_m^{n+1}\right) - V_m\psi_m^{n+1}\bigg\} = \\
		&= \psi_m^n + \frac{i\tau}{2}\bigg\{\frac{1}{2h^2}\left(\psi_{m+1}^n +
		\psi_{m-1}^n - 2\psi_m^n\right) - V_m\psi_m^n\bigg\}
\end{align*}

ugotovimo, da je ekvivalenten sistemu

\begin{equation}
	L_+ \mathbf{\Psi}^{(n+1)} = L_- \mathbf{\Psi}^{(n)} = \mathbf{b}^{(n)},
\end{equation}

kjer je matrika $L_+$ enaka

\begin{equation}
	L_\pm = 1 \mp \frac{i\tau}{4h^2}\begin{bmatrix}
		a_0 & 1 & 0 & \ldots & & & 0\\
		1 & a_1 & 1 & 0 & \ldots & & 0\\
		0 & 1 & a_2 & 1 & 0 & \ldots & 0\\
		\vdots & & & & \ddots \\
		& & & \ldots & 0 & 1 & a_{M-1} \end{bmatrix}, \qquad
	a_k = -2 - 2h^2V_k,
\end{equation}

vektorji $\mathbf{\Psi}^{(n)}$ pa so stolpci

\begin{equation}
	\mathbf{\Psi}^{(n)} = \begin{bmatrix} \psi_0^n \\ \psi_1^n \\ \vdots \\ \psi_{M-1}^n
		\end{bmatrix}.
\end{equation}

Količin z indeksom $n+1$ ne poznamo, od koder se vidi iteracijski značaj naše metode. Problem
Schr\" odingerjeve enačbe smo tako prevedli na reševanje tridiagonalnega sistema.

Videti je, da metodo dobimo poceni, vendar moramo časovno stabilnost drago plačati, zaradi
pogoja $h^2 \gg \tau$. To lahko izboljšamo s tem, da drugi odvod aproksimiramo z višjim redom
diferenc, npr. petim redom. V tem primeru imamo pasato matriko s petimi "`diagonalami"'.

Naša enačba se potem glasi

\begin{align*}
	&\psi_m^{n+1} - \frac{i\tau}{2}\bigg\{\frac{1}{2h^2}\left[\frac{4}{3}\left(
		\psi_{m+1}^{n+1} + \psi_{m-1}^{n+1}\right) - \frac{1}{12}\left(
		\psi_{m+2}^{n+1} + \psi_{m-2}^{n+1}\right) - \frac{5}{2}\psi_m^{n+1}\right] 
		- V_m\psi_m^{n+1}\bigg\} = \\
	&= \psi_m^n + \frac{i\tau}{2}\bigg\{\frac{1}{2h^2}\left[\frac{4}{3}\left(
		\psi_{m+1}^n + \psi_{m-1}^n\right) - \frac{1}{12}\left(
		\psi_{m+2}^n + \psi_{m-2}^n\right) - \frac{5}{2}\psi_m^n\right] -
		V_m\psi_m^n\bigg\},
\end{align*}

v tem primeru si lahko privoščimo nižje med razmerje $\tau$ in $h^2$. V tem primeru
dobimo

\begin{equation}
	L_\pm = 1 \mp \frac{i\tau}{4h^2}\begin{bmatrix}
		a_0 & 4/3 & -1/12 & 0 & \ldots & \ldots & 0 \\
		4/3 & a_1 & 4/3 & -1/12 & 0 & \ldots & 0 \\
		-1/12 & 4/3 & a_2 & 4/3 & -1/12 & \ldots & 0 \\
		\vdots & & & & \ddots & \\
		0 & \ldots & \ldots & 0 & -1/12 & 4/3 & a_{M-1} \end{bmatrix}, \qquad
	a_k = -\frac{5}{2} - 2h^2V_m,
\end{equation}

kar pride zelo prav, saj je na primer dobro razmerje že $\tau = 0.01$ in $h = 0.1$ za prvih
$\sim 1000$ iteracij.

\section{Rezultati}

\end{document}

